\documentclass[11pt,a4paper]{article}
\usepackage{ctex}

% base.tex - 基础配置文件
\usepackage[utf8]{inputenc}
\usepackage[T1]{fontenc}
\usepackage{amsmath,amsfonts,amssymb}
\usepackage{graphicx}
\usepackage{float}
\usepackage{geometry}
\usepackage{xcolor}
\usepackage{framed}
\usepackage{tcolorbox}
\usepackage{tikz}
\usepackage{pgfplots}
\usepackage{enumitem}

\geometry{left=2.5cm,right=2.5cm,top=2.5cm,bottom=2.5cm}

% 定义颜色
\definecolor{formalcolor}{RGB}{240,248,255}
\definecolor{problemcolor}{RGB}{248,248,255}
\definecolor{jiexicolor}{RGB}{250,250,250}

% 定义标题命令
\newcommand{\mytitle}[1]{
    \begin{center}
        {\LARGE\textbf{#1}}
    \end{center}
    \vspace{1cm}
}

% 定义formal环境
\newenvironment{formal}
{
    \begin{tcolorbox}[colback=formalcolor,colframe=blue!50!black,title=知识要点]
}
{
    \end{tcolorbox}
}

% 定义problem环境
\newcounter{problemcounter}
\newenvironment{problem}
{
    \stepcounter{problemcounter}
    \begin{tcolorbox}[colback=problemcolor,colframe=black!20!white,title=例题 \arabic{problemcounter}]
}
{
    \end{tcolorbox}
    \vspace{0.5cm}
}

% 定义jiexi环境
\newenvironment{jiexi}[1][0]
{
    \begin{tcolorbox}[colback=jiexicolor,colframe=green!50!black,title=解析]
    \textbf{解:}\\
}
{
    \end{tcolorbox}
}

% 定义其他常用命令
\newcommand{\classinfo}[4]{
    % 课程信息命令,暂时为空实现
}

% 页面设置
\pagestyle{plain}

% 数学符号设置
\DeclareMathOperator{\lcm}{lcm}
% \gcd already defined, so we skip it

% 定义hmwk环境
\newcounter{hmwkcounter}
\newenvironment{hmwk}
{
    \stepcounter{hmwkcounter}
    \begin{tcolorbox}[colback=white,colframe=blue!50!black,title=作业 \arabic{hmwkcounter}]
}
{
    \end{tcolorbox}
    \vspace{0.5cm}
}

% 定义\sclear命令
\newcommand{\sclear}{\clearpage}

% 其他自定义命令
\newcommand{\abs}[1]{\left|#1\right|}
\newcommand{\norm}[1]{\left\|#1\right\|}

\graphicspath{ {pictures/} }
\begin{document}


\maketitle

\mytitle{第8课\quad 基本不等式及其应用}

% \classinfo{G1}{基本不等式及其应用}{掌握基本不等式的基本应用方法以及解题逻辑;掌握三角形不等式的应用方法}{基本不等式}

\begin{formal}
    {\large \textbf{知识点一、平均值不等式}}
\end{formal}


\begin{enumerate}
    \item 对于正数$a,b$称$\displaystyle \frac{a+b}{2}\text{是}a,b\text{的算数平均值,并称}\sqrt{ab}\text{是}a,b\text{的几何平均值}$

    \item {\large{\textbf{平均值不等式:}}}\ 两个正数的算术平均值大于等于它们的几何平均值,即对于任意的正数,有
    $$
    \frac{a+b}{2}\ge \sqrt{ab},\ (a>0,b>0)
    $$
    当且仅当$a=b$时等号成立\\
    
    {\large{\textbf{三注意:}}}

    {\large{\textbf{“一正”:}}}不等式中的各项必须都是正数; \\
    {\large{\textbf{“二定”:}}}和定积最大,积定和最小; \\
    {\large{\textbf{“三相等”:}}}只有满足了不等式中等号成立的条件,才能使用基本不等式求最值.\\
    \item 平均值不等式的变式与推广
    \begin{enumerate}
        \item 调和平均数$\le$ 几何平均数$\le$ 算术平均数$\le$平方平均数:
        % \item 几何平均数$\le$ 算术平均数$\le$平方平均数:
        $$\displaystyle \frac{2}{\frac{1}{a}+\frac{1}{b}} \le \sqrt{ab} \le \frac{a+b}{2}  \le  \sqrt{\frac{a^2+b^2}{2}}, \ (a>0,b>0,\text{当且仅当$a=b$时取等号})$$
        % $$\displaystyle \sqrt{ab} \le \frac{a+b}{2}  \le  \sqrt{\frac{a^2+b^2}{2}}, \ (a>0,b>0,\text{当且仅当$a=b$时取等号})$$
    
        \item { 推广:$ a_1,a_2,a_3,\cdots,a_n$是$n$个正数,则$\displaystyle \frac{a_1,a_2,a_3,\cdots,a_n}{n}$称为这$n$个正数的算术平均数,\\ 
        $\sqrt[n]{a_1\cdot a_2\cdot a_3\cdot \cdots \cdot a_n}$称为这个正数的几何平均数,
        它们的关系是:
        $$\displaystyle \frac{a_1,a_2,a_3,\cdots,a_n}{n} \ge \sqrt[n]{a_1\cdot a_2\cdot a_3\cdot \cdots \cdot a_n}$$
        当且仅当$  a_1=a_2=a_3=\cdots=a_n$时等号成立。}
    \end{enumerate}
\end{enumerate}
\clearpage
\begin{tcolorbox} 
    \centering
    题型一:基本不等式基本应用
    \end{tcolorbox}
\begin{problem} 
设 \(\displaystyle x > 0\) ,证明 \(\displaystyle x + \frac{1}{x} \geq  2\) ,并指出等号成立条件
\begin{jiexi}[25]
证明 因为 \(\displaystyle x > 0\) ,由平均值不等式,得

\[
x + \frac{1}{x} \geq  2\sqrt{x \cdot  \frac{1}{x}} = 2,
\]

且等号只有当 \(\displaystyle x = \frac{1}{x}\) ,即 \(\displaystyle {x}^{2} = 1\) 时才成立. 由于 \(\displaystyle x > 0\) ,所以 \(\displaystyle x = 1\) .

因此,当且仅当 \(\displaystyle x = 1\) 时, \(\displaystyle x + \frac{1}{x} = 2\) .


\end{jiexi}
\end{problem}
\begin{problem} 
    证明:若 \(\displaystyle x < 0\) ,则 \(\displaystyle x + \frac{1}{x} \leq   - 2\) ,并指出等号成立条件。
    \begin{jiexi}[25]
 证明 根据题意, \(\displaystyle a < 0\) ,则 \(\displaystyle - a > 0\) ,

左式 \(\displaystyle = a + \frac{1}{a} =  - \left\lbrack  {\left( {-a}\right)  + \left( {-\frac{1}{a}}\right) }\right\rbrack\) ,

又由 \(\displaystyle \left( {-a}\right)  + \left( {-\frac{1}{a}}\right)  \geq\)

\(\displaystyle 2\sqrt{\left( {-a}\right)  \times  \left( {-\frac{1}{a}}\right) } = 2,\)

则有 \(\displaystyle a + \frac{1}{a} \leq   - 2\) ,当且仅当 \(\displaystyle a =  - 1\) 时,等

号成立. 故 \(\displaystyle a + \frac{1}{a} \leq   - 2\) ,当且仅当 \(\displaystyle a =  - 1\)

时, 等号成立.


\end{jiexi}
\end{problem}
\begin{problem} \(\displaystyle x \neq  0,x \in  R\) ,则 \(\displaystyle x + \frac{1}{x}\) 的取值范围\_\_\_\_\_;
    \begin{jiexi}
【答案】(1) \(\displaystyle \left( {-\infty , - 2\rbrack \cup \lbrack 2, + \infty }\right)\) ;


\end{jiexi}
\end{problem}
\begin{problem} 
    设 \(\displaystyle {ab} > 0\) ,证明 \(\displaystyle \frac{b}{a} + \frac{a}{b} \geq  2\) ,并指出等号成立的条件
    \begin{jiexi}[25]
证明 因为 \(\displaystyle {ab} > 0\) ,所以 \(\displaystyle a\) 、 \(\displaystyle b\) 同号,因而 \(\displaystyle \frac{b}{a} > 0,\frac{a}{b} > 0\) .

由平均值不等式, 得

\[
\frac{b}{a} + \frac{a}{b} \geq  2\sqrt{\frac{b}{a} \cdot  \frac{a}{b}} = 2,
\]

且等号当且仅当 \(\displaystyle \frac{b}{a} = \frac{a}{b}\) ,即 \(\displaystyle a = b\) 时才成立.


\end{jiexi}
\end{problem}
\begin{problem} 

    已知 \(\displaystyle a + b = 1,a,b \in  \mathrm{R}\) ,求证: \(\displaystyle {a}^{2} + {b}^{2} \geq  \frac{1}{2}\) ,并指出等号成立的条件.
    \begin{jiexi}[25]
常用不等式: \(\displaystyle 2\left( {{a}^{2} + {b}^{2}}\right)  \geq  {\left( a + b\right) }^{2}\)

解 对任意给定的实数 \(\displaystyle a,b\) ,总有 \(\displaystyle {a}^{2} + {b}^{2} \geq  {2ab}\) ,且等号当且仅当 \(\displaystyle a = b\) 时

成立.

两边同时加上 \(\displaystyle {a}^{2} + {b}^{2}\) ,得 \(\displaystyle 2\left( {{a}^{2} + {b}^{2}}\right)  \geq  {\left( a + b\right) }^{2}\) ,即 \(\displaystyle {a}^{2} + {b}^{2} \geq  \frac{{\left( a + b\right) }^{2}}{2} = \frac{1}{2}\) .

等号当且仅当 \(\displaystyle a = b = \frac{1}{2}\) 时成立.

注意 \(\displaystyle 2\left( {{a}^{2} + {b}^{2}}\right)  \geq  {\left( a + b\right) }^{2}\) 也是我们常用的一个不等式. 
\end{jiexi}
\end{problem}
\begin{problem} 

(6)若对任意 \(\displaystyle a > 0\) , \(\displaystyle b > 0\) ,不等式 \(\displaystyle \frac{2}{a} + \frac{1}{b} \geq  \frac{m}{{2a} + b}\) 恒成立,则 \(\displaystyle m\) 的取值范围是\_\_\_\_\_. 
\begin{jiexi}
【答案】 \(\displaystyle \left( {-\infty ,9}\right)\) . 

【解答】解: \(\displaystyle \because a > 0,b > 0,\therefore {2a} + b > 0,\because\) 不等式 \(\displaystyle \frac{2}{a} + \frac{1}{b} \geq  \frac{m}{{2a} + b}\) 恒成立, \(\displaystyle \therefore m \leq  \frac{2\left( {{2a} + b}\right) }{a} + \frac{{2a} + b}{b} = 5 + \frac{2b}{a} + \frac{2a}{b}\) 恒成立,

\(\displaystyle \because \frac{2b}{a} + \frac{2a}{b} \geq  2\sqrt{\frac{2b}{a} \cdot  \frac{2a}{b}} = 4\) ,当且仅当 \(\displaystyle \frac{2b}{a} = \frac{2a}{b}\) ,即 \(\displaystyle a = b\) 时取等号, \(\displaystyle \therefore m \leq  9\) ,即 \(\displaystyle m \in  ( - \infty ,9\rbrack\)

\end{jiexi}
\end{problem}
\begin{problem} 


设 \(\displaystyle x \in  R\) ,求函数 \(\displaystyle y = x\left( {4 - x}\right)\) 的最大值
\begin{jiexi}[25]
【答案】 4


\end{jiexi}
\end{problem}
\begin{problem} 
设 \(\displaystyle a\) 、 \(\displaystyle b\) 为正数,且 \(\displaystyle a + {2b} = 1\) ,比较 \(\displaystyle {ab}\) 与 \(\displaystyle \frac{1}{8}\) 的值的大小
\begin{jiexi}[25]
【答案】 \(\displaystyle {ab} < \frac{1}{8}\)

\end{jiexi}
\end{problem}
\begin{problem} 


已知 \(\displaystyle y = {2x}\sqrt{1 - {x}^{2}}\left( {0 < x < 1}\right)\) ,求 \(\displaystyle y\) 的最大值
\begin{jiexi}[25]
【答案】 1


\end{jiexi}
\end{problem}
\begin{problem} 
若 \(\displaystyle {x}^{2} + {y}^{2} = 1\) ,则 \(\displaystyle {xy}\) 的取值范围.
\begin{jiexi}[25]
【答案】 \(\displaystyle \left\lbrack  {-\frac{1}{2},\frac{1}{2}}\right\rbrack\)


\end{jiexi}
\end{problem}
\begin{problem} 
已知 \(\displaystyle a,b \in  {R}^{ + }\) , \(\displaystyle {a}^{2} + \frac{{b}^{2}}{2} = 1\) ,求 \(\displaystyle a\sqrt{1 + {b}^{2}}\) 的最大值。
\begin{jiexi}[25]
【答案】 \(\displaystyle \frac{\sqrt{2}}{2}\)

【解析】 \(\displaystyle a\sqrt{1 + {b}^{2}} = \sqrt{2} \cdot  \left( {a \cdot  \sqrt{\frac{1 + {b}^{2}}{2}}}\right)  \leq  \sqrt{2}\left( \frac{{a}^{2} + \frac{1 + {b}^{2}}{2}}{2}\right)  = \sqrt{2} \cdot  \frac{1}{2} = \frac{\sqrt{2}}{2}\)

当且仅当 \(\displaystyle a = \sqrt{\frac{1 + {b}^{2}}{2}}\) 时取等,因此 \(\displaystyle a\sqrt{1 + {b}^{2}}\) 的最大值为 \(\displaystyle \frac{\sqrt{2}}{2}\) 

\end{jiexi}
\end{problem}


\begin{problem} 
若 \(\displaystyle 0 < x < 1,0 < y < 1\) ,且 \(\displaystyle x \neq  y\) ,求在 \(\displaystyle {x}^{2} + {y}^{2},{2xy},x + y,2\sqrt{xy}\) 中的最大数和最小数
\begin{jiexi}[25]
【答案】【 \(\displaystyle x + y\) 】

分析 可先用平均值不等式作判断,以减少比较的次数.

解 由平均值不等式,知 \(\displaystyle {x}^{2} + {y}^{2} > {2xy},x + y > 2\sqrt{xy}\) .

又 \(\displaystyle \because 0 < x < 1,0 < y < 1\) ,

\(\displaystyle \therefore {x}^{2} < x,{y}^{2} < y\) .

\(\displaystyle \therefore {x}^{2} + {y}^{2} < x + y\) ,故最大数为 \(\displaystyle x + y\) .

而 \(\displaystyle x < \sqrt{x},y < \sqrt{y}\) ,

\(\displaystyle \therefore {2xy} < 2\sqrt{xy}\) ,故最小数为 \(\displaystyle {2xy}\) .


\end{jiexi}
\end{problem}
\begin{problem} 
已知 \(\displaystyle a,b\) 均为正实数,求证: \(\displaystyle \frac{2}{\frac{1}{a} + \frac{1}{b}} \leq  \sqrt{ab} \leq  \frac{a + b}{2} \leq  \sqrt{\frac{{a}^{2} + {b}^{2}}{2}}\) ,并指出等号成立的条件
\begin{jiexi}[25]
\(\displaystyle \sqrt{\frac{{a}^{2} + {b}^{2}}{2}}\) . 分析 由平均值不等式知 \(\displaystyle \frac{a + b}{2} \geq  \sqrt{ab}\) ,故本题只需证明 \(\displaystyle \frac{2}{\frac{1}{a} + \frac{1}{b}} \leq  \sqrt{ab}\) 和 \(\displaystyle \frac{a + b}{2} \leq\) 又易证 \(\displaystyle 2\left( {{a}^{2} + {b}^{2}}\right)  \geq  {\left( a + b\right) }^{2}.\therefore \frac{{a}^{2} + {b}^{2}}{2} \geq  \frac{{\left( a + b\right) }^{2}}{4}\) .

证明 \(\displaystyle \because a,b\) 均为正实数, \(\displaystyle \therefore \frac{1}{a} > 0,\frac{1}{b} > 0\) . \(\displaystyle \therefore \sqrt{\frac{{a}^{2} + {b}^{2}}{2}} \geq  \frac{a + b}{2}\) ,且等号当且仅当 \(\displaystyle a = b\) 时成立. 由平均值不等式,得 \(\displaystyle \frac{\frac{1}{a} + \frac{1}{b}}{2} \geq  \sqrt{\frac{1}{ab}} > 0\) . 而由平均值不等式,得 \(\displaystyle \sqrt{ab} \leq  \frac{a + b}{2}\) 显然成立,且等号当且仅当 \(\displaystyle a = b\) 时成

\(\displaystyle \therefore \frac{2}{\frac{1}{a} + \frac{1}{b}} \leq  \sqrt{ab}\) ,且等号当且仅当 \(\displaystyle a = b\) 时成立. 立. \(\displaystyle \therefore \frac{2}{\frac{1}{a} + \frac{1}{b}} \leq  \sqrt{ab} \leq  \frac{a + b}{2} \leq  \sqrt{\frac{{a}^{2} + {b}^{2}}{2}}\) .


\end{jiexi}
\end{problem}
\begin{problem} 
如果正数 \(\displaystyle a,b,c,d\) 满足 \(\displaystyle a + b = {cd} = 4\) ,那么(   )

(A) \(\displaystyle {ab} \leq  c + d\) ,且等号成立时 \(\displaystyle a,b,c,d\) 的取值唯一

(B) \(\displaystyle {ab} \geq  c + d\) ,且等号成立时 \(\displaystyle a,b,c,d\) 的取值唯一

(C) \(\displaystyle {ab} \leq  c + d\) ,且等号成立时 \(\displaystyle a,b,c,d\) 的取值不唯一

(D) \(\displaystyle {ab} \geq  c + d\) ,且等号成立时 \(\displaystyle a,b,c,d\) 的取值不唯一
\begin{jiexi}
【答案】A

如果 \(\displaystyle a\) , \(\displaystyle b\) 是正数,则根据均值不等式有: \(\displaystyle a + b \geq  2\sqrt{ab}\)

,则 \(\displaystyle {\left( a + b\right) }^{2} \geq  {4ab}\)

如果 \(\displaystyle c\) , \(\displaystyle d\) 是正数,则根据均值不等式有: \(\displaystyle c + d \geq  2\sqrt{cd}\)

;则 \(\displaystyle {cd} \leq  \frac{{\left( c + d\right) }^{2}}{4}\)

\(\displaystyle \because a,b,c,d\) 满足 \(\displaystyle a + b = {cd} = 4\) ,

\(\displaystyle \therefore 2\sqrt{ab} \leq  a + b = {cd} \leq  \frac{{\left( c + d\right) }^{2}}{4}\)

当且仅当 \(\displaystyle a = b = c = d = 2\) 时取等号.

化简即为: \(\displaystyle {ab} \leq  c + d\) 且等号成立时 \(\displaystyle a\) , \(\displaystyle b\) , \(\displaystyle c\) , \(\displaystyle d\) 的取值唯

-.

故选: \(\displaystyle A\) .

 (1) \(\displaystyle x + \frac{1}{x + a} \Rightarrow  x + a + \frac{1}{x + a} - a\)

 (2) \(\displaystyle x\left( {a - {bx}}\right)  \Rightarrow  {bx}\left( {a - {bx}}\right)  \cdot  \frac{1}{b}\)

\end{jiexi}
\end{problem}

\clearpage
\begin{tcolorbox} 
    \centering
    题型二:配凑法
    \tcblower %增加了一条虚线
    $$x+\frac{1}{x+a} \quad \Rightarrow \quad x+a+\frac{1}{x+a}-a$$
    $$x(a-bx) \quad \Rightarrow \quad bx(a-bx)\cdot \frac{1}{b} $$
    \end{tcolorbox}
\begin{problem} 
函数 \(\displaystyle y = x + \frac{1}{x - 1}\left( {x > 1}\right)\) 的最小值为\_\_\_\_\_
\begin{jiexi}
【答案】 3


\end{jiexi}
\end{problem}
\begin{problem} 
已知 \(\displaystyle 0 < x < \frac{1}{2}\) ,求 \(\displaystyle y = \frac{1}{2}x\left( {1 - {2x}}\right)\) 的最大值为\_\_\_\_\_
\begin{jiexi}
【答案】 \(\displaystyle \frac{1}{16}\)


\end{jiexi}
\end{problem}
\begin{problem} 
已知 \(\displaystyle x < \frac{5}{4}\) ,求函数 \(\displaystyle y = {4x} - 2 + \frac{1}{{4x} - 5}\) 的最大值。
\begin{jiexi}[25]
【答案】因 \(\displaystyle {4x} - 5 < 0\) ,所以首先要 “调整” 符号,又 \(\displaystyle \left( {{4x} - 2}\right)  \cdot  \frac{1}{{4x} - 5}\) 不是常数,

所以对 \(\displaystyle {4x} - 2\) 要进行拆、凑项, \(\displaystyle \because x < \frac{5}{4},\therefore 5 - {4x} > 0\) ,

\(\displaystyle \therefore y = {4x} - 2 + \frac{1}{{4x} - 5} =  - \left( {5 - {4x} + \frac{1}{5 - {4x}}}\right)  + 3 \leq   - 2 + 3 = 1\)

当且仅当 \(\displaystyle 5 - {4x} = \frac{1}{5 - {4x}}\) ,即 \(\displaystyle x = 1\) 时,上式等号成立,故当 \(\displaystyle x = 1\) 时, \(\displaystyle {y}_{\max } = 1\) 。


\end{jiexi}
\end{problem}
\begin{problem} 
已知 \(\displaystyle x > \frac{3}{2},x + \frac{4}{{2x} - 3}\) 的最小值为\_\_\_\_\_.
\begin{jiexi}
【解析】(2) \(\displaystyle x + \frac{4}{{2x} - 3} = x + \frac{2}{x - \frac{3}{2}} = x - \frac{3}{2} + \frac{2}{x - \frac{3}{2}} + \frac{3}{2} \geq  2\sqrt{2} + \frac{3}{2}\) ,当且仅当 \(\displaystyle x - \frac{3}{2} = \frac{2}{x - \frac{3}{2}}\) 时取等.


\end{jiexi}
\end{problem}

\begin{tcolorbox} 
    \centering
    题型三:换元法
    \tcblower %增加了一条虚线
    \begin{enumerate}
        \item 观察形式,型如$\displaystyle y=\frac{ex+f}{ax^2+bx+c}$或$\displaystyle y=\frac{ax^2+bx+c}{ex+f};(\frac{\text{一次}}{\text{二次}}\text{或}\frac{\text{二次}}{\text{一次}})$
        \item 令$t=ex+f$对式子进行配凑成$\displaystyle y=at+\frac{b}{t}$的形式,(对一次式进行整体换元)
        \item 再利用基本不等式求最值
    \end{enumerate}
\end{tcolorbox}

\begin{problem} 
已知 \(\displaystyle x > 0\) ,求函数 \(\displaystyle y = \frac{{x}^{2} + x + 1}{x}\) 的最小值
\begin{jiexi}[25]
【答案】 8


\end{jiexi}
\end{problem}
\begin{problem} 
已知 \(\displaystyle x >  - 1\) ,求函数 \(\displaystyle y = \frac{{x}^{2} + x + 1}{x + 1}\) 的最小值
\begin{jiexi}[25]
【答案】 1


\end{jiexi}
\end{problem}
\begin{problem} 
已知 \(\displaystyle x >  - 2\) ,求函数 \(\displaystyle y = \frac{x + 2}{{x}^{2} - x + 6}\) 的最小值
\begin{jiexi}[25]
【答案】 \(\displaystyle \frac{4\sqrt{3} + 5}{23}\)


\end{jiexi}
\end{problem}
\begin{problem} 
若 \(\displaystyle x >  - 1\) ,则函数 \(\displaystyle y = \frac{x + 1}{{x}^{2} + {3x} + 3}\) 的最大值为\_\_\_\_\_.
\begin{jiexi}
【答案】 \(\displaystyle \frac{1}{3}\)


\end{jiexi}
\end{problem}
\begin{problem} 
函数 \(\displaystyle y = \frac{6\sqrt{{x}^{2} + 2}}{{x}^{2} + 4}\) 的最大值为\_\_\_\_\_.
\begin{jiexi}
【答案】 \(\displaystyle \frac{3\sqrt{2}}{2}\)


\end{jiexi}
\end{problem}
\begin{problem} 
(1)已知 \(\displaystyle a,b\) 均为正数,则 \(\displaystyle \frac{a + {2b}}{a} + \frac{a + b}{b}\) 的最小值是\_\_\_\_\_;

(2)已知 \(\displaystyle a,b\) 均为正数,则 \(\displaystyle \frac{a}{a + {2b}} + \frac{b}{a + b}\) 的最小值是\_\_\_\_\_.
\begin{jiexi}
【难度】★★

【答案】(1) \(\displaystyle 2\sqrt{2} + 2\) ;(2) \(\displaystyle 2\sqrt{2} - 2\)

【解析】( 1 ) \(\displaystyle \frac{a + {2b}}{a} + \frac{a + b}{b} = 1 + \frac{2b}{a} + \frac{a}{b} + 1 \geq  2 + 2\sqrt{2}\) ,当且仅当 \(\displaystyle a = \sqrt{2}b\) 时取等.

(2)解法一:设 \(\displaystyle \left\{  \begin{array}{l} a + {2b} = x \\  a + b = y \end{array}\right.\) ,则 \(\displaystyle \left\{  \begin{array}{l} a = {2y} - x \\  b = x - y \end{array}\right.\) ,代入原式可得: \(\displaystyle \frac{{2y} - x}{x} + \frac{x - y}{y} = \frac{2y}{x} + \frac{x}{y} - 2 \geq  2\sqrt{2} - 2\) ,

当且仅当 \(\displaystyle x = \sqrt{2}y\) 时取等.

解法二: \(\displaystyle \frac{a}{a + {2b}} + \frac{b}{a + b} = \frac{{a}^{2} + {2ab} + 2{b}^{2}}{{a}^{2} + {3ab} + 2{b}^{2}} = 1 - \frac{ab}{{a}^{2} + {3ab} + {b}^{2}} = 1 - \frac{1}{\frac{a}{b} + \frac{2b}{a} + 3} \geq  2\sqrt{2} - 2\) ,

当且仅当 \(\displaystyle \frac{a}{b} = \frac{2b}{a}\) ,即 \(\displaystyle b = \sqrt{2}a\) 时等号成立.


\end{jiexi}
\end{problem}
\begin{tcolorbox} 
    \centering
    题型四:“1”的代换(齐次化)
    \tcblower %增加了一条虚线
    题目已知$x+y=m$,求$\displaystyle \frac{a}{x}+\frac{b}{y}$\quad 或\quad 已知$\displaystyle \frac{a}{x}+\frac{b}{y}=m$,求$x+y$时\\
    (1)把已知条件变成“1” \\
    (2)两式乘起来,用基本不等式
\end{tcolorbox}

\begin{problem} 
设 \(\displaystyle a > 0,b > 0\) ,且 \(\displaystyle a + b = 1\) ,则 \(\displaystyle \frac{1}{a} + \frac{1}{b}\) 的最小值为\_\_\_\_\_
\begin{jiexi}
【答案】 4


\end{jiexi}
\end{problem}
\begin{problem} 
设 \(\displaystyle a > 0,b > 0\) ,且 \(\displaystyle a + b = 2\) ,则 \(\displaystyle \frac{1}{a} + \frac{1}{b}\) 的最小值为\_\_\_\_\_
\begin{jiexi}
【答案】 2


\end{jiexi}
\end{problem}
\begin{problem} 
设 \(\displaystyle a > 0,b > 0\) ,且 \(\displaystyle \frac{3}{a} + \frac{4}{b} = 1\) ,则 \(\displaystyle a + b\) 的最小值为\_\_\_\_\_
\begin{jiexi}
【答案】 \(\displaystyle 4\sqrt{3} + 7\)


\end{jiexi}
\end{problem}
\begin{problem} 
设 \(\displaystyle a > 0,b > 0\) ,且 \(\displaystyle a + b = 2\) ,则 \(\displaystyle \frac{1}{a} + \frac{1}{b - 1}\) 的最小值为\_\_\_\_\_
\begin{jiexi}
【答案】 4

【注意】题目已知 \(\displaystyle x + y = m\) ,求 \(\displaystyle \frac{a}{x} + \frac{b}{y}\) 配凑时未知数的一致性


\end{jiexi}
\end{problem}
\begin{problem} 
若 \(\displaystyle x\) 、 \(\displaystyle y\) 为正实数满足 \(\displaystyle x + {3y} = {5xy}\) ,求 \(\displaystyle {3x} + {4y}\) 的最小值
\begin{jiexi}[25]
【答案】 5


\end{jiexi}
\end{problem}
\begin{problem} 
已知 \(\displaystyle a > 0,b > 0\) , \(\displaystyle \frac{1}{a + 1} + \frac{1}{b + 1} = 1\) ,则 \(\displaystyle \mathrm{a} + 2\mathrm{\;b}\) 的最小值为\_\_\_\_\_
\begin{jiexi}
【答案】 5


\end{jiexi}
\end{problem}
\begin{problem} 
设 \(\displaystyle x > 0,y > 0\) ,且 \(\displaystyle x + y = 1\) ,则 \(\displaystyle \frac{1}{{2x} + y} + \frac{2}{y + 3}\) 的最小值为\_\_\_\_\_
\begin{jiexi}
【答案】 \(\displaystyle \frac{3 + 2\sqrt{2}}{5}\)


\end{jiexi}
\end{problem}
\begin{problem} 
设 \(\displaystyle x > 0,y > 0\) ,且 \(\displaystyle {2x} + {3y} = 5\) ,则 \(\displaystyle \frac{1}{x + y} + \frac{2}{y + 3}\) 的最小值为\_\_\_\_\_
\begin{jiexi}
【答案】 1 

\end{jiexi}
\end{problem}

\begin{problem} 
若 \(\displaystyle A,B,C\) 为 \(\displaystyle \triangle  {ABC}\) 的三个内角,则 \(\displaystyle \frac{4}{A} + \frac{1}{B + C}\) 的最小值为\_\_\_\_\_. 
\begin{jiexi}

【答案】 \(\displaystyle \frac{9}{\pi }\) (3) \(\displaystyle A + B + C = \pi  \Rightarrow  \frac{4}{A} + \frac{1}{B + C} = \left( {\frac{4}{A} + \frac{1}{B + C}}\right)  \cdot  \left( {A + B + C}\right)  \cdot  \frac{1}{\pi } = \frac{1}{\pi }\left( {5 + \frac{4\left( {B + C}\right) }{A} + \frac{A}{B + C}}\right)  \geq  \frac{9}{\pi }\) ,当且

仅当 \(\displaystyle A = 2\left( {B + C}\right)\) 时取等.


\end{jiexi}
\end{problem}
\begin{problem} 
已知 \(\displaystyle 0 < x < 1\) ,函数 \(\displaystyle y = \frac{1}{x} + \frac{2}{1 - x}\) 的最小值为\_\_\_\_\_
\begin{jiexi}
【备注】 本题中1的代换相对隐晦,不妨这样理解,如令 \(\displaystyle {x}^{\prime } = \frac{1}{x},{y}^{\prime } = \frac{1}{1 - x}\) ,问题就转化

为: \(\displaystyle {x}^{\prime },{y}^{\prime } > 0\) , \(\displaystyle \frac{1}{{x}^{\prime }} + \frac{1}{{y}^{\prime }} = 1\) ,求 \(\displaystyle {x}^{\prime } + 2{y}^{\prime }\) 的最小值 .

【解析】 \(\displaystyle \because 0 < x < 1\) ,

\(\displaystyle \therefore 0 < 1 - x < 1\) ,

\(\displaystyle y = \frac{1}{x} + \frac{2}{1 - x} = \left( {\frac{1}{x} + \frac{2}{1 - x}}\right) \left( {x + 1 - x}\right)\)

\(\displaystyle = 3 + \frac{1 - x}{x} + \frac{2x}{1 - x} \geq  3 + 2\sqrt{2},\)

当且仅当 \(\displaystyle \frac{1 - x}{x} = \frac{2x}{1 - x}\) ,

即 \(\displaystyle x = \sqrt{2} - 1\) 时,等号成立.


\end{jiexi}
\end{problem}
\begin{problem} 
设 \(\displaystyle a\) 、 \(\displaystyle b\) 是正实数,且 \(\displaystyle a + {2b} = 2\) ,则 \(\displaystyle \frac{{a}^{2}}{a + 1} + \frac{4{b}^{2}}{{2b} + 1}\) 的最小值是\_\_\_\_\_.
\begin{jiexi}
【答案】 1

【解析】( 1 ) \(\displaystyle \frac{{a}^{2}}{a + 1} + \frac{4{b}^{2}}{{2b} + 1} = \frac{1}{4}\left( {\frac{{a}^{2}}{a + 1} + \frac{4{b}^{2}}{{2b} + 1}}\right)  \cdot  \left( {a + 1 + {2b} + 1}\right)\)

\(\displaystyle = \frac{1}{4}\left\lbrack  {{a}^{2} + 4{b}^{2} + \frac{{a}^{2}\left( {{2b} + 1}\right) }{a + 1} + \frac{4{b}^{2}\left( {a + 1}\right) }{{2b} + 1}}\right\rbrack   \geq  \frac{1}{4}\left( {{a}^{2} + 4{b}^{2} + {4ab}}\right)  = \frac{1}{4}{\left( a + 2b\right) }^{2} = 1,\)

当且仅当 \(\displaystyle a\left( {{2b} + 1}\right)  = {2b}\left( {a + 1}\right)\) 时取等.


\end{jiexi}
\end{problem}
\begin{problem} 
设 \(\displaystyle a + b = {2019},b > 0\) ,则当 \(\displaystyle a =\) \_\_\_\_\_时, \(\displaystyle \frac{1}{{2019}\left| a\right| } + \frac{\left| a\right| }{b}\) 取的最小值。
\begin{jiexi}[45]
【难度】 \(\displaystyle \star   \star   \star\)

【答案】 \(\displaystyle - \frac{2019}{2018}\)

【解析】 \(\displaystyle \frac{1}{{2019}\left| a\right| } + \frac{\left| a\right| }{b} = \frac{a + b}{{2019}^{2}\left| a\right| } + \frac{\left| a\right| }{b} = \frac{a}{{2019}^{2}\left| a\right| } + \frac{b}{{2019}^{2}\left| a\right| } + \frac{\left| a\right| }{b}\)

\(\displaystyle \geq   - \frac{1}{{2019}^{2}} + \frac{b}{{2019}^{2}\left| a\right| } + \frac{\left| a\right| }{b} \geq   - \frac{1}{{2019}^{2}} + 2\sqrt{\frac{1}{{2019}^{2}}}\) ,

当且仅当 \(\displaystyle \frac{b}{{2019}^{2}\left| a\right| } = \frac{\left| a\right| }{b}\) 即 \(\displaystyle b =  - {2019a}\) 时等号成立, \(\displaystyle a + b = {2019}\) ,则 \(\displaystyle a =  - \frac{2019}{2018}\) .


\end{jiexi}
\end{problem}
\begin{problem} 
非零实数 \(\displaystyle x\) 、 \(\displaystyle y\) 、 \(\displaystyle z\) 满足 \(\displaystyle {x}^{2} + {y}^{2} + {z}^{2} = 1\) ,则 \(\displaystyle \frac{1}{{x}^{2}} + \frac{1}{{y}^{2}} + \frac{1}{{z}^{2}}\) 的最小值是\_\_\_\_\_。
\begin{jiexi}
【难度】★★

【解析】1的妙用, 可以从局部和整体妙用1, 这也是针对于这类问题的基本思路。答案是9


\end{jiexi}
\end{problem}
\begin{problem} 
若正数 \(\displaystyle a,b\) 满足 \(\displaystyle \frac{1}{a} + \frac{1}{b} = 1,\frac{1}{a - 1} + \frac{9}{b - 1}\) 的最小值为\_\_\_\_\_。
\begin{jiexi}[55]
【答案】 6

【解答】解: \(\displaystyle \because\) 正数 \(\displaystyle a,b\) 满足 \(\displaystyle \frac{1}{a} + \frac{1}{b} = 1,\therefore a > 1\) ,且 \(\displaystyle b > 1\) ;

\(\displaystyle \frac{1}{a} + \frac{1}{b} = 1\) 变形为 \(\displaystyle \frac{a + b}{ab} = 1,\therefore {ab} = a + b,\therefore {ab} - a - b = 0,\therefore \left( {a - 1}\right) \left( {b - 1}\right)  = 1,\therefore a - 1 = \frac{1}{b - 1}\) ;

\(\displaystyle \therefore a - 1 > 0,\therefore \frac{1}{a - 1} + \frac{9}{b - 1} = \frac{1}{a - 1} + 9\left( {a - 1}\right)  \geq  2\sqrt{\frac{1}{a - 1} \cdot  9\left( {a - 1}\right) } = 6\) ,

当且仅当 \(\displaystyle \frac{1}{a - 1} = 9\left( {a - 1}\right)\) ,即 \(\displaystyle a = 1 \pm  \frac{1}{3}\) 时取 “ \(\displaystyle =\) ” (由于 \(\displaystyle a > 1\) ,故取 \(\displaystyle a = \frac{4}{3}\) ),

\(\displaystyle \therefore \frac{1}{a - 1} + \frac{9}{b - 1}\) 的最小值为 6 ;

\end{jiexi}
\end{problem}
\begin{tcolorbox} 
    \centering
    题型五:消元法
    \tcblower %增加了一条虚线
    (1)写:根据问题形式写出基本不等式\\
    (2)换:求谁留谁,把另一个通过条件整体替换掉\\
    (3)解:将$ab$或$a+b$视作整体,解二次不等式
\end{tcolorbox}

\begin{problem} 
设 \(\displaystyle a,b\) 为正实数,且 \(\displaystyle {ab} = a + b + 3\) ,则求 \(\displaystyle {ab}\) 和 \(\displaystyle a + b\) 的取值范围
\begin{jiexi}[25]
【答案】 \(\displaystyle \lbrack 9, + \infty )\;\lbrack 6, + \infty )\)


\end{jiexi}
\end{problem}
\begin{problem} 
若 \(\displaystyle x\) , \(\displaystyle y\) 为正实数,满足 \(\displaystyle x + {2y} + {2xy} = 8\) ,则求 \(\displaystyle x + {2y}\) 的最小值
\begin{jiexi}[25]
【答案】 8


\end{jiexi}
\end{problem}
\begin{problem} 
若 \(\displaystyle x\) , \(\displaystyle y\) 为正实数,满足 \(\displaystyle {2x} + {3y} + {xy} = 6\) ,则求 \(\displaystyle {2xy}\) 的最小值
\begin{jiexi}[25]
【答案】 0


\end{jiexi}
\end{problem}
\begin{problem} 
已知 \(\displaystyle a > 0\) , \(\displaystyle b > 0\) ,当 \(\displaystyle {\left( a + 4b\right) }^{2} + \frac{1}{ab}\) 取到最小值时, \(\displaystyle b =\) \_\_\_\_\_;
\begin{jiexi}
【难度】★★★

【答案】(1) \(\displaystyle \frac{1}{4}\) ;(2)16

【解析】( 1 )由 \(\displaystyle {\left( a + 4b\right) }^{2} + \frac{1}{ab} \geq  {\left( 2\sqrt{a \cdot  {4b}}\right) }^{2} + \frac{1}{ab} = {16ab} + \frac{1}{ab} \geq  2\sqrt{16} = 8\) ,

当且仅当 \(\displaystyle \left\{  {\begin{array}{l} a = {4b} \\  {16ab} = \frac{1}{ab} \end{array} \Rightarrow  \left\{  \begin{array}{l} a = 1 \\  b = \frac{1}{4} \end{array}\right. }\right.\) 时取等号.


\end{jiexi}
\end{problem}
\begin{problem} 
设 \(\displaystyle a > b > 0\) ,求 \(\displaystyle {a}^{2} + \frac{16}{b\left( {a - b}\right) }\) 的最小值.

\begin{jiexi}[45]
    解法一: \(\displaystyle {a}^{2} + \frac{16}{b\left( {a - b}\right) } \geq  {a}^{2} + \frac{16}{{\left\lbrack  \frac{b + \left( {a - b}\right) }{2}\right\rbrack  }^{2}} = {a}^{2} + \frac{64}{{a}^{2}} \geq  {16}\) ,

两次等号同时成立条件: \(\displaystyle \left\{  {\begin{array}{l} b = \left( {b - a}\right) \\  {a}^{2} = \frac{64}{{a}^{2}} \end{array} \Rightarrow  \left\{  \begin{array}{l} a = 2\sqrt{2} \\  b = \sqrt{2} \end{array}\right. }\right.\) ;

解法二: \(\displaystyle {a}^{2} + \frac{16}{b\left( {a - b}\right) } = {\left\lbrack  b + \left( a - b\right) \right\rbrack  }^{2} + \frac{16}{b\left( {a - b}\right) } = \left\lbrack  {{b}^{2} + {\left( a - b\right) }^{2}}\right\rbrack   + \left\lbrack  {{2b}\left( {a - b}\right)  + \frac{16}{b\left( {a - b}\right) }}\right\rbrack\)

\(\displaystyle \geq  {2b}\left( {a - b}\right)  + \left\lbrack  {{2b}\left( {a - b}\right)  + \frac{16}{b\left( {a - b}\right) }}\right\rbrack   = {4b}\left( {a - b}\right)  + \frac{16}{b\left( {a - b}\right) } \geq  {16},\)

两次等号同时成立条件: \(\displaystyle \left\{  {\begin{array}{l} b = \left( {b - a}\right) \\  {4b}\left( {a - b}\right)  = \frac{16}{b\left( {a - b}\right) } \end{array} \Rightarrow  \left\{  \begin{array}{l} a = 2\sqrt{2} \\  b = \sqrt{2} \end{array}\right. }\right.\) .


\end{jiexi}
\end{problem}
\begin{problem} 
当 \(\displaystyle 0 < x < a\) 时,不等式 \(\displaystyle \frac{1}{{x}^{2}} + \frac{1}{{\left( a - x\right) }^{2}} \geq  2\) 恒成立,求 \(\displaystyle a\) 的最大值.
\begin{jiexi}[45]
【难度】★★

【答案】 2

【解析】方法一: (基本不等式) 由 \(\displaystyle \frac{1}{{x}^{2}} + \frac{1}{{\left( a - x\right) }^{2}} \geq  2\sqrt{\frac{1}{{x}^{2}{\left( a - x\right) }^{2}}} = \frac{2}{x\left( {a - x}\right) } \geq  \frac{2}{\frac{{\left\lbrack  x + \left( a - x\right) \right\rbrack  }^{2}}{4}} = \frac{8}{{a}^{2}}\) ,

当且仅当 \(\displaystyle x = \frac{a}{2}\) 时,取到最小值. 即 \(\displaystyle {\left( \frac{1}{{x}^{2}} + \frac{1}{{\left( a - x\right) }^{2}}\right) }_{\min } = \frac{8}{{a}^{2}}\) 恒成立.

所以 \(\displaystyle \frac{8}{{a}^{2}} \geq  2\) ,则 \(\displaystyle a \leq  2\) ,故实数 \(\displaystyle a\) 的最大值为 2 .

方法二:(构造齐次式)

\[
\left( {\frac{1}{{x}^{2}} + \frac{1}{{\left( a - x\right) }^{2}}}\right)  \cdot  {a}^{2} = \left( {\frac{1}{{x}^{2}} + \frac{1}{{\left( a - x\right) }^{2}}}\right)  \cdot  {\left\lbrack  x + \left( a - x\right) \right\rbrack  }^{2} = \left( {\frac{1}{{x}^{2}} + \frac{1}{{\left( a - x\right) }^{2}}}\right)  \cdot  \left\lbrack  {{x}^{2} + 2\left( {a - x}\right) x + {\left( a - x\right) }^{2}}\right\rbrack
\]

\(\displaystyle = 2 + \left\lbrack  {\frac{2x}{\left( a - x\right) } + \frac{2\left( {a - x}\right) }{x}}\right\rbrack   + \left\lbrack  {\frac{{\left( a - x\right) }^{2}}{{x}^{2}} + \frac{{x}^{2}}{{\left( a - x\right) }^{2}}}\right\rbrack   \geq  2 + 4 + 2 = 8\) ,即 \(\displaystyle {\left( \frac{1}{{x}^{2}} + \frac{1}{{\left( a - x\right) }^{2}}\right) }_{\min } = \frac{8}{{a}^{2}}\) ,同上.


\end{jiexi}
\end{problem}

\begin{tcolorbox} 
    \centering
    题型六:基本不等式的应用
\end{tcolorbox}

\begin{problem} 
(1)证明:(1)周长为常数的所有矩形中正方形的面积最大

( 2 )面积相同的所有矩形中正方形的周长最小.
\begin{jiexi}[55]
证明(1)设矩形的周长为常数 \(\displaystyle l > 0\) ,而其长、宽分别为 \(\displaystyle x\text{ 、 }y > 0\) ,就有 \(\displaystyle {2x} + {2y} = l\) . 此矩形的面积为 \(\displaystyle S = {xy}\) . 由平均值不等式, 有

\[
S = {xy} \leq  {\left( \frac{x + y}{2}\right) }^{2} = {\left( \frac{l}{4}\right) }^{2},
\]

当且仅当 \(\displaystyle x = y = \frac{l}{2}\) ,即矩形为正方形时,面积 \(\displaystyle S\) 取得最大值 \(\displaystyle \frac{{l}^{2}}{16}\) .

(2)设矩形的面积为常数 \(\displaystyle S > 0\) ,而其长、宽仍分别设为 \(\displaystyle x\text{ 、 }y > 0\) ,就有 \(\displaystyle {xy} = S\) . 此矩形的周长为 \(\displaystyle l = 2\left( {x + y}\right)\) . 由平均值不等式, 有

\[
\frac{x + y}{2} \geq  \sqrt{xy} = \sqrt{S},
\]

所以 \(\displaystyle l = 2\left( {x + y}\right)  \geq  4\sqrt{S}\) ,且当且仅当 \(\displaystyle x = y = \sqrt{S}\) ,即矩形为正方形时,周长 \(\displaystyle l\) 取最小值 \(\displaystyle 4\sqrt{S}\) .


\end{jiexi}
\end{problem}
\sclear
\begin{problem} 
在城市旧城改造中,某小区为了升级居住环境,拟在小区的闲置地中规划一个面积为 \(\displaystyle {200}{\mathrm{\;m}}^{2}\) 的矩形区域 (如图所示),按规划要求: 在矩形内的四周安排 \(\displaystyle {2m}\) 宽的绿化,绿化造价为 \(\displaystyle {200}\mathrm{元}/{\mathrm{m}}^{2}\) ,中间区域地面硬化以方便后期放置各类健身器材,硬化造价为 \(\displaystyle {100}\mathrm{元}/{\mathrm{m}}^{2}\) . 设矩形的长为 \(\displaystyle x\left( m\right)\) .


\includegraphics[max width=0.3\textwidth]{images/01968fa5-d2e3-7182-940c-34b4aef08981_12_254_442_426_307_0.jpg}

 

( 1 )设总造价 \(\displaystyle y\) (元)表示为长度 \(\displaystyle x\left( \mathrm{m}\right)\) 的函数;

(2)当 \(\displaystyle x\left( \mathrm{\;m}\right)\) 取何值时,总造价最低,并求出最低总造价.
\begin{jiexi}[65]
【答案】(1) \(\displaystyle y = {18400} + {400}\left( {x + \frac{200}{x}}\right) ,x \in  \left( {4,{50}}\right)\) ;

( 2 )当 \(\displaystyle x = {10}\sqrt{2}\) 时,总造价最低为 \(\displaystyle {18400} + {8000}\sqrt{2}\) 元.

【解析】(1)由矩形的长为 \(\displaystyle x\left( \mathrm{\;m}\right)\) ,则矩形的宽为 \(\displaystyle \frac{200}{x}\left( \mathrm{\;m}\right)\) ,

则中间区域的长为 \(\displaystyle x - 4\left( \mathrm{\;m}\right)\) ,宽为 \(\displaystyle \frac{200}{x} - 4\left( \mathrm{\;m}\right)\) ,则定义域为 \(\displaystyle x \in  \left( {4,{50}}\right)\) ,

则 \(\displaystyle y = {100} \times  \left\lbrack  {\left( {x - 4}\right) \left( {\frac{200}{x} - 4}\right) }\right\rbrack   + {200}\left\lbrack  {{200} - \left( {x - 4}\right) \left( {\frac{200}{x} - 4}\right) }\right\rbrack\) ,

整理得 \(\displaystyle y = {18400} + {400}\left( {x + \frac{200}{x}}\right) ,x \in  \left( {4,{50}}\right)\) .

(2) \(\displaystyle x + \frac{200}{x} \geq  2\sqrt{x \cdot  \frac{200}{x}} = {20}\sqrt{2}\) ,当且仅当 \(\displaystyle x = \frac{200}{x}\) 时取等号,即 \(\displaystyle x = {10}\sqrt{2} \in  \left( {4,{50}}\right)\) ,

所以当 \(\displaystyle x = {10}\sqrt{2}\) 时,总造价最低为 \(\displaystyle {18400} + {8000}\sqrt{2}\) 元.


\end{jiexi}
\end{problem}
\begin{problem} 
某汽车运输公司,购买了一批豪华大客车投入运营,据市场分析每辆客车运营总利润 \(\displaystyle y\) (单位: 10 万元)与运营年数 \(\displaystyle x\left( {x \in  N}\right)\) 为二次函数关系,则每辆客车运营多少年,其运营的年平均利润最大? 并求最大年平均利润。


\includegraphics[max width=0.3\textwidth]{images/01968fa5-d2e3-7182-940c-34b4aef08981_13_975_392_436_315_0.jpg}

\begin{jiexi}[25]

【难度】★

【答案】 \(\displaystyle \mathrm{x} = 5\) ,最大年平均利润是 20 万元。


\end{jiexi}
\end{problem}
\sclear
\begin{problem} 
某单位用木料制作如图所示的框架,框架的下部是边长分别为 \(\displaystyle x\) 、 \(\displaystyle y\) (单位: \(\displaystyle m\) )的矩形. 上部是等腰直角三角形. 要求框架围成的总面积 \(\displaystyle 8{\mathrm{\;{cm}}}^{2}\) . 问 \(\displaystyle \mathrm{x}\text{ 、 }\mathrm{y}\) 分别为多少 (精确到 \(\displaystyle {0.001}\mathrm{\;m}\) ) 时用料最省? 
\begin{jiexi}[45]
【答案】由题意得

\(\displaystyle \mathrm{{xy}} + \frac{1}{4}{\mathrm{x}}^{2} = 8,\therefore \mathrm{y} = \frac{8 - \frac{{x}^{2}}{4}}{x} = \frac{8}{x} - \frac{x}{4}\left( {0 < \mathrm{x} < 4\sqrt{2}}\right) .\)


\includegraphics[max width=0.2\textwidth]{images/01968fa5-d2e3-7182-940c-34b4aef08981_13_1050_874_299_443_0.jpg}

 

于定, 框架用料长度为

\(\displaystyle 1 = {2x} + {2y} + 2\left( {\frac{\sqrt{2}}{2}x}\right)  = \left( {\frac{3}{2} + \sqrt{2}}\right) x + \frac{16}{x} \geq  4\sqrt{6 + 4\sqrt{2}}.\)

当 \(\displaystyle \left( {\frac{3}{2} + \sqrt{2}}\right) \mathrm{x} = \frac{16}{x}\) ,即 \(\displaystyle \mathrm{x} = 8 - 4\sqrt{2}\) 时等号成立.

此时, \(\displaystyle \mathrm{x} \approx  {2.343},\mathrm{y} = 2\sqrt{2} \approx  {2.828}\) .

故当 \(\displaystyle \mathrm{x}\) 为 \(\displaystyle {2.343}\mathrm{\;m}\) , \(\displaystyle \mathrm{y}\) 为 \(\displaystyle {2.828}\mathrm{\;m}\) 时,用料最省.


\end{jiexi}
\end{problem}
\begin{problem} 
在面积为 \(\displaystyle \pi\) 的圆中作一个内接矩形,使它的面积最大. 求此矩形面积的最大值及此时矩形的各边长.
\begin{jiexi}[55]
【答案】 2, \(\displaystyle \sqrt{2},\sqrt{2}\)

\(\displaystyle \because\) 圆的面积为 \(\displaystyle \pi\)

\(\displaystyle \therefore\) 圆的半径 \(\displaystyle r = 1\)  \(\displaystyle \therefore y = 2\sqrt{{1}^{2} - {\left( \frac{x}{2}\right) }^{2}}\) 设矩形宽为 \(\displaystyle x\left( {0 < x < 2}\right)\) ,长为

\(\displaystyle y\left( {0 < y < 2}\right)\) 面积

\(\displaystyle \because\) 圆的半径的平方 \(\displaystyle =\) 宽的一半的平方+长的一 \(\displaystyle s = {2x}\sqrt{1 - {\left( \frac{x}{2}\right) }^{2}} = \sqrt{4{x}^{2} - {x}^{4}}\) 半的平方

,即 \(\displaystyle {r}^{2} = {\left( \frac{x}{2}\right) }^{2} + {\left( \frac{y}{2}\right) }^{2}\)  \(\displaystyle = \sqrt{-{\left( {x}^{2} - 2\right) }^{2} + 4}\)

\(\displaystyle \therefore \frac{y}{2} = \sqrt{{1}^{2} - {\left( \frac{x}{2}\right) }^{2}}\) 当 \(\displaystyle {x}^{2} = 2\) 时,即 \(\displaystyle x = \sqrt{2}\) , \(\displaystyle s\) 有最大值2 矩形长 \(\displaystyle = \sqrt{2}\) ,宽 \(\displaystyle = \sqrt{2}\)


\end{jiexi}
\end{problem}
\begin{tcolorbox} 
    \centering
题型七:综合题
\end{tcolorbox}


\begin{problem} 
(1)设实数 \(\displaystyle x,y\) 满足 \(\displaystyle {2x} + y = 1\) .

(I)求 \(\displaystyle 4{x}^{2} + {y}^{2} + {3xy}\) 的最小值;

(II)若 \(\displaystyle x > 0,y > 0\) ,求 \(\displaystyle \frac{1}{x} + \frac{2}{y} - \sqrt{2xy}\) 的最小值.
\begin{jiexi}[50]
【答案】 \(\displaystyle \frac{7}{8}\) .

【解答】解:(I)因为 \(\displaystyle {2x} + y = 1\) ,则 \(\displaystyle y = 1 - {2x}\) ,

所以 \(\displaystyle 4{x}^{2} + {y}^{2} + {3xy} = 4{x}^{2} + {\left( 1 - 2x\right) }^{2} + {3x}\left( {1 - {2x}}\right)  = 4{x}^{2} + 1 + 4{x}^{2} - {4x} + {3x} - 6{x}^{2} = 2{x}^{2} - x + 1 = 2\left( {{x}^{2} - \frac{1}{2}x}\right)  + 1 =\)  \(\displaystyle 2{\left( x - \frac{1}{4}\right) }^{2} - \frac{1}{8} + 1,\) 当 \(\displaystyle x = \frac{1}{4}\) 时, \(\displaystyle 2{\left( x - \frac{1}{4}\right) }^{2} + \frac{7}{8}\) 取得最小值,最小值为 \(\displaystyle \frac{7}{8}\) , 所以当 \(\displaystyle x = \frac{1}{4},y = \frac{1}{2}\) 时, \(\displaystyle 4{x}^{2} + {y}^{2} + {3xy}\) 取得最小值 \(\displaystyle \frac{7}{8}\) . 

(II)因为 \(\displaystyle x > 0,y > 0,{2x} + y = 1\) , 所以 \(\displaystyle \left( {\frac{1}{x} + \frac{2}{y}}\right)  = \left( {\frac{1}{x} + \frac{2}{y}}\right) \left( {{2x} + y}\right)  = 4 + \frac{y}{x} + \frac{4x}{y} \geq  4 + 2\sqrt{\frac{y}{x} \cdot  \frac{4x}{y}} = 4\) , 当且仅当 \(\displaystyle \frac{y}{x} = \frac{4x}{y}\) ,即 \(\displaystyle {2x} = y = \frac{1}{2}\) 时取等号, 又因为 \(\displaystyle {2x} + y \geq  2\sqrt{{2x} \cdot  y} = 2\sqrt{2xy}\) , 所以 \(\displaystyle - \sqrt{2xy} \geq   - \frac{{2x} + y}{2} =  - \frac{1}{2}\) ,当且仅当 \(\displaystyle {2x} = y = \frac{1}{2}\) 时等号成立, 所以 \(\displaystyle \frac{1}{x} + \frac{2}{y} - \sqrt{2xy} \geq  8 - \frac{1}{2} = \frac{15}{2}\) ,当且仅当 \(\displaystyle {2x} = y = \frac{1}{2}\) 时等号成立, 所以 \(\displaystyle \frac{1}{x} + \frac{2}{y} - \sqrt{2xy}\) 的最小值为 \(\displaystyle \frac{15}{2}\) . 


\end{jiexi}
\end{problem}
\sclear
\begin{problem} 
若实数 \(\displaystyle x\text{ 、 }y\text{ 、 }m\) 满足 \(\displaystyle \left| {x - m}\right|  < \left| {y - m}\right|\) ,则称 \(\displaystyle x\) 比 \(\displaystyle y\) 更接近 \(\displaystyle m\) . 

(1)若4比 \(\displaystyle \left( {{x}^{2} - {3x}}\right)\) 更接近 0,求 \(\displaystyle x\) 的取值范围; 

(2)对任意两个不相等的正数 \(\displaystyle a\) 、 \(\displaystyle b\) ,判断并证明: \(\displaystyle \left( {a + b}\right)\) 和 \(\displaystyle \left( {\frac{{b}^{2}}{a} + \frac{{a}^{2}}{b}}\right)\) 哪个更接近 \(\displaystyle 2\sqrt{ab}\) 
\begin{jiexi}[55]
 (1) 由题意,得 \(\displaystyle \left| {{x}^{2} - {3x}}\right|  > 4,\therefore {x}^{2} - {3x} > 4\) 或 \(\displaystyle {x}^{2} - {3x} <  - 4\) . \(\displaystyle \therefore x \in  \left( {-\infty , - 1}\right)  \cup  \left( {4, + \infty }\right) ;\)
 
 \(\displaystyle \;\left( 2\right) \because a > 0,b > 0\) ,且 \(\displaystyle a \neq  b,\therefore a + b > 2\sqrt{ab}\) , \(\displaystyle \frac{{b}^{2}}{a} + \frac{{a}^{2}}{b} > 2\sqrt{ab}\) . \(\displaystyle \therefore \left| {\frac{{b}^{2}}{a} + \frac{{a}^{2}}{b} - 2\sqrt{ab}}\right|  - \left| {a + b - 2\sqrt{ab}}\right|  = \left( {\frac{{b}^{2}}{a} + \frac{{a}^{2}}{b} - 2\sqrt{ab}}\right)  - (a + b -\)  \(\displaystyle 2\sqrt{ab}) = \frac{{b}^{2}}{a} + \frac{{a}^{2}}{b} - \left( {a + b}\right)  = \frac{{b}^{2} - {ab}}{a} + \frac{{a}^{2} - {ab}}{b} = \frac{b}{a}\left( {b - a}\right)  + \frac{a}{b}\left( {a - b}\right)  = (a -\) b) \(\displaystyle \left( {\frac{a}{b} - \frac{b}{a}}\right)  = \frac{\left( {a - b}\right) \left( {{a}^{2} - {b}^{2}}\right) }{ab} = \frac{{\left( a - b\right) }^{2}\left( {a + b}\right) }{ab} > 0\) ,则 \(\displaystyle \left| {\frac{{b}^{2}}{a} + \frac{{a}^{2}}{b} - 2\sqrt{ab}}\right|  >  \mid  a + b -\)  \(\displaystyle 2\sqrt{ab} \mid  .\therefore \left( {a + b}\right)\) 比 \(\displaystyle \left( {\frac{{b}^{2}}{a} + \frac{{a}^{2}}{b}}\right)\) 更接近 \(\displaystyle 2\sqrt{ab}\)


\end{jiexi}
\end{problem}
\begin{problem} 
若 \(\displaystyle a > 0,b > 0\) ,且 \(\displaystyle a + b = 4\) ,则下列不等式恒成立的是(   )

A. \(\displaystyle {a}^{2} + {b}^{2} \geq  8\) B. \(\displaystyle \frac{1}{ab} \geq  \frac{1}{4}\) C. \(\displaystyle \sqrt{a} + \sqrt{b} \leq  2\sqrt{2}\) D. \(\displaystyle \frac{1}{a} + \frac{1}{b} \leq  1\)
\begin{jiexi}[25]
【答案】 \(\displaystyle {ABC}\) .

【解答】解: \(\displaystyle A\) . 因为 \(\displaystyle a > 0,b > 0\) ,且 \(\displaystyle a + b = 4\) ,所以 \(\displaystyle b = 4 - a\) ,所以 \(\displaystyle {a}^{2} + {b}^{2} = {a}^{2} + {\left( 4 - a\right) }^{2} = 2{a}^{2} - {8a} + {16}\) , 根据二次函数性质可知,当 \(\displaystyle a = b = 2\) 时, \(\displaystyle {a}^{2} + {b}^{2}\) 取最小值 8,故有 \(\displaystyle {a}^{2} + {b}^{2} \geq  8\) 成立, \(\displaystyle A\) 正确; \(\displaystyle B\) . 因为 \(\displaystyle a > 0,b > 0,a + b = 4\) ,所以 \(\displaystyle {ab} \leq  {\left( \frac{a + b}{2}\right) }^{2} = 4\) ,当且仅当 \(\displaystyle a = b = 2\) 时取等号, 所以 \(\displaystyle \frac{1}{ab} \geq  \frac{1}{4}\) ,故 \(\displaystyle B\) 正确. \(\displaystyle C\) : 因为 \(\displaystyle {\left( \sqrt{a} + \sqrt{b}\right) }^{2} = a + b + 2\sqrt{ab} \leq  a + b + a + b = 8\) ,当且仅当 \(\displaystyle a = b = 2\) 时取等号, 所以 \(\displaystyle \sqrt{a} + \sqrt{b} \leq  2\sqrt{2},C\) 正确; \(\displaystyle D : \frac{1}{a} + \frac{1}{b} = \frac{1}{4}\left( {\frac{a + b}{a} + \frac{a + b}{b}}\right)  = \frac{1}{4}\left( {2 + \frac{b}{a} + \frac{a}{b}}\right)  \geq  \frac{1}{4}\left( {2 + 2}\right)  = 1\) ,当且仅当 \(\displaystyle a = b = 2\) 时取等号, \(\displaystyle D\) 错误.
\end{jiexi}
\end{problem}
\begin{problem} 
下列说法正确的有(   ) 

A. \(\displaystyle y = \frac{{x}^{2} + 1}{x}\) 的最小值为 2 

B. 已知 \(\displaystyle x > 1\) ,则 \(\displaystyle y = {2x} + \frac{4}{x - 1} - 1\) 的最小值为 \(\displaystyle 4\sqrt{2} + 1\) 

C. 若正数 \(\displaystyle x,y\) 为实数,若 \(\displaystyle x + {2y} = {3xy}\) ,则 \(\displaystyle {2x} + y\) 的最大值为 3 

D. 设 \(\displaystyle x,y\) 为实数,若 \(\displaystyle 9{x}^{2} + {y}^{2} + {xy} = 1\) ,则 \(\displaystyle {3x} + y\) 的最大值为 \(\displaystyle \frac{2\sqrt{21}}{7}\) 
\begin{jiexi}
【答案】 \(\displaystyle {BD}\) . 【解答】解: 对于 \(\displaystyle A\) ,当 \(\displaystyle x < 0\) 时, \(\displaystyle y < 0\) ,故 \(\displaystyle A\) 错误,对于 \(\displaystyle B\) ,当 \(\displaystyle x > 1\) 时, \(\displaystyle x - 1 > 0\) , \(\displaystyle \therefore y = {2x} + \frac{4}{x - 1} - 1 = 2\left( {x - 1}\right)  + \frac{4}{x - 1} + 1 \geq  2\sqrt{8} + 1 = 4\sqrt{2} + 1\) ,当且仅当 \(\displaystyle x = \sqrt{2} + 1\) 时,等号成立,故 \(\displaystyle B\) 正确,对于 \(\displaystyle C\) ,若正数 \(\displaystyle x\text{ 、 }y\) 满足 \(\displaystyle x + {2y} = {3xy}\) ,则 \(\displaystyle \frac{2}{x} + \frac{1}{y} = 3\) ,

\(\displaystyle \therefore {2x} + y = \frac{1}{3}\left( {{2x} + y}\right) \left( {\frac{2}{x} + \frac{1}{y}}\right)  = \frac{1}{3}\left( {\frac{2x}{y} + \frac{2y}{x} + 5}\right)  \geq  \frac{1}{3}\left( {2\sqrt{4} + 5}\right)  = 3\) ,当且仅当 \(\displaystyle x = y = 1\) 时,等号成立, 故 \(\displaystyle C\) 错误,

对于 \(\displaystyle D,1 = 9{x}^{2} + {y}^{2} + {xy} = 9{x}^{2} + {y}^{2} + {6xy} - {5xy} = {\left( 3x + y\right) }^{2} - \frac{5}{3} \cdot  3{x}^{ * }y \geq  {\left( 3x + y\right) }^{2} - \frac{5}{3} \cdot  \frac{{\left( 3x + y\right) }^{2}}{4} = \frac{7}{12}{\left( 3x + y\right) }^{2}\) , 所以 \(\displaystyle {\left( 3x + y\right) }^{2} \leq  \frac{12}{7}\) ,可得 \(\displaystyle - \frac{2\sqrt{21}}{7} \leq  {3x} + y \leq  \frac{2\sqrt{21}}{7}\) ,当且仅当 \(\displaystyle y = {3x}\) 时,等号成立,故 \(\displaystyle {3x} + y\) 的最大值为 \(\displaystyle \frac{2\sqrt{21}}{7}\) , 故 \(\displaystyle D\) 正确.



\end{jiexi}
\end{problem}

\clearpage
\begin{formal}
    {\large \textbf{知识点二、三角不等式}}
\end{formal}
三角不等式:两个实数和的绝对值小于等于他们绝对值的和,
即对于任意给定的实数$a,b$,有$$|a+b|\le|a|+|b|$$且等号当且仅当$ab\ge0$时成立\\
如果$a,b$是实数,那么

\begin{enumerate}
    \item $||a|-|b||\le|a+b|\le |a|+|b|$且左等号当且仅当$ab\le0$时成立;且右等号当且仅当$ab\ge0$时成立
    \item $||a|-|b||\le|a-b|\le |a|+|b|$且左等号当且仅当$ab\ge0$时成立;且右等号当且仅当$ab\le0$时成立
\end{enumerate}

推论:$|a_1+a_2+\cdots+a_n|\le |a_1|+|a_2|+\cdots+|a_n|$,当且仅当$ab\ge0$时成立


\begin{tcolorbox} 
    \centering
    综合题
    \tcblower %增加了一条虚线
    使用三角不等式或者前面学的绝对值不等式解题
\end{tcolorbox}

\begin{problem} 
(1)写出不等式 \(\displaystyle \left| {x + y}\right|  \leq  \left| x\right|  + \left| y\right|\) 等号成立的一个充要条件是\_\_\_\_\_,

一个充分非必要条件是\_\_\_\_\_,

一个必要非充分条件是\_\_\_\_\_;

(2)写出不等式 \(\displaystyle \left| x\right|  - \left| y\right|  \leq  \left| {x - y}\right|\) 等号成立的一个充要条件是\_\_\_\_\_;

( 3 )写出不等式 \(\displaystyle \left| x\right|  - \left| y\right|  \leq  \left| {x + y}\right|\) 等号成立的一个充要条件是\_\_\_\_\_;

( 4 )写出不等式 \(\displaystyle \left| {x - y}\right|  \leq  \left| x\right|  + \left| y\right|\) 等号成立的一个充要条件是\_\_\_\_\_.
\begin{jiexi}
【答案】(1)“ \(\displaystyle {xy} \geq  0\) ”或“ \(\displaystyle x = {ky},k > 0\) ”;“ \(\displaystyle x = 0\) ”;“ \(\displaystyle {xy} \geq   - 1\) ”

(2) “ \(\displaystyle {xy} \geq  {y}^{2}\) ” (3) “ \(\displaystyle {xy} + {y}^{2} \leq  0\) ” (4)“ \(\displaystyle {xy} \leq  0\) ”(答案不唯一)

\end{jiexi}
\end{problem}
\begin{problem} 
代数式 \(\displaystyle y = \left| {x - 4}\right|  + \left| {x - 6}\right|\) 的最小值为\_\_\_\_\_
\begin{jiexi}
【答案】 2

【解析】 \(\displaystyle y = \left| {x - 4}\right|  + \left| {x - 6}\right|  \geq  \left| {x - 4 + 6 - x}\right|  = 2\) .

\end{jiexi}
\end{problem}
\begin{problem} 
已知 \(\displaystyle \left| x\right|  < \frac{a}{4},\left| y\right|  < \frac{a}{6}\) . 求证: \(\displaystyle \left| {{2x} - {3y}}\right|  < a\) .
\begin{jiexi}[35]
【答案】证明 \(\displaystyle \because \left| x\right|  < \frac{a}{4},\left| y\right|  < \frac{a}{6},\therefore \left| {2x}\right|  < \frac{a}{2},\left| {3y}\right|  < \frac{a}{2}\) ,

根据绝对值三角不等式可得 \(\displaystyle \left| {{2x} - {3y}}\right|  \leq  \left| {2x}\right|  + \left| {3y}\right|  < \frac{a}{2} + \frac{a}{2} = a\) .


\end{jiexi}
\end{problem}
\begin{problem} 

已知 \(\displaystyle \left| {x - a}\right|  < \frac{c}{2},\left| {y - b}\right|  < \frac{c}{2}\) ,求证 \(\displaystyle \left| {\left( {x + y}\right)  - \left( {a + b}\right) }\right|  < c\) .
\begin{jiexi}[35]
【答案】证明 \(\displaystyle \left| {\left( {x + y}\right)  - \left( {a + b}\right) }\right|  = \left| {\left( {x - a}\right)  + \left( {y - b}\right) }\right|  \leq  \left| {x - a}\right|  + \left| {y - b}\right| \;\left( 1\right)\)

\(\displaystyle \because \left| {x - a}\right|  < \frac{c}{2},\left| {y - b}\right|  < \frac{c}{2},\therefore \left| {x - a}\right|  + \left| {y - b}\right|  < \frac{c}{2} + \frac{c}{2} = c\) (2)

由 (1),(2) 得: \(\displaystyle \left| {\left( {x + y}\right)  - \left( {a + b}\right) }\right|  < c\)

\end{jiexi}
\end{problem}
\begin{problem} 
\(\displaystyle x\) 为实数,且 \(\displaystyle \left| {x - 5}\right|  + \left| {x - 3}\right|  < m\) 有解,则 \(\displaystyle m\) 的取值范围是\_\_\_\_\_
\begin{jiexi}
【答案】 \(\displaystyle m > 2\)
\end{jiexi}
\end{problem}
\begin{problem}
已知关于 \(\displaystyle x\) 的不等式 \(\displaystyle \left| {x + 1}\right|  - \left| {x - 2}\right|  > t\) 有解,则实数 \(\displaystyle t\) 的取值范围是\_\_\_\_\_.
\begin{jiexi}
【答案】 \(\displaystyle t < 3\)

\end{jiexi}
\end{problem}
\begin{problem} 
若存在实数 \(\displaystyle x\) ,使得 \(\displaystyle \left| {x - 1}\right|  + \left| {x + 2}\right|  < a\) 成立,则实数 \(\displaystyle a\) 的取值范围为\_\_\_\_\_.
\begin{jiexi}
【答案】 \(\displaystyle a > 3\)

\end{jiexi}
\end{problem}
\begin{problem} 

    若不等式 \(\displaystyle \left| {x - 2}\right|  + \left| {x + 3}\right|  > a\) ,对于 \(\displaystyle x \in  \mathbf{R}\) 均成立,那么实数 \(\displaystyle a\) 的取值范围是\_\_\_\_\_.
    \begin{jiexi}
【答案】 \(\displaystyle \left( {-\infty ,5}\right)\)


\end{jiexi}
\end{problem}
\begin{problem} 
不等式 \(\displaystyle \left| {x - a}\right|  + \left| {x - 3}\right|  > 4\) 对一切实数 \(\displaystyle x\) 恒成立,求实数 \(\displaystyle a\) 的取值范围.
\begin{jiexi}
【答案】 \(\displaystyle a > 7\) 或 \(\displaystyle a <  - 1\)

【详解】因为 \(\displaystyle \left| {x - a}\right|  + \left| {x - 3}\right|  = \left| {x - a}\right|  + \left| {3 - x}\right|  \geq  \left| {x - a + 3 - x}\right|  = \left| {3 - a}\right|\) ,当且仅当 \(\displaystyle x - a = 3 - x\) 时取等号,

即当 \(\displaystyle x = \frac{3 + a}{2}\) 时取等号,所以 \(\displaystyle \left| {x - a}\right|  + \left| {x - 3}\right|\) 最小值是 \(\displaystyle \left| {3 - a}\right|\) ,要想不等式 \(\displaystyle \left| {x - a}\right|  + \left| {x - 3}\right|  > 4\) 对一切实数

\(\displaystyle x\) 恒成立,只需 \(\displaystyle \left| {3 - a}\right|  > 4\) ,解得 \(\displaystyle a > 7\) 或 \(\displaystyle a <  - 1\) .


\end{jiexi}
\end{problem}
\begin{problem} 
存在实数 \(\displaystyle x\) ,使得不等式 \(\displaystyle \left| {x + 3}\right|  + \left| {x - 1}\right|  \leq  {a}^{2} - {3a}\) 有解,则实数 \(\displaystyle a\) 的取值范围为\_\_\_\_\_.
\begin{jiexi}
【答案】 \(\displaystyle \left( {-\infty , - 1\rbrack \cup \lbrack 4, + \infty }\right)\) .

【详解】由绝对值不等式的性质,可得 \(\displaystyle \left| {x + 3}\right|  + \left| {x - 1}\right|  \geq  \left| {\left( {x + 3}\right)  - \left( {x - 1}\right) }\right|  = 4\) ,当且仅当

\(\displaystyle \left( {x + 3}\right)  \cdot  \left( {x - 1}\right)  \leq  0\) 时等号成立,要使得不等式 \(\displaystyle \left| {x + 3}\right|  + \left| {x - 1}\right|  \leq  {a}^{2} - {3a}\) 有解,转化为

\(\displaystyle {\left( \left| x + 3\right|  + \left| x - 1\right| \right) }_{\min } \leq  {a}^{2} - {3a}\) 恒成立,

所以 \(\displaystyle {a}^{2} - {3a} \geq  4\) ,解得 \(\displaystyle a \geq  4\) 或 \(\displaystyle a \leq   - 1\) ,即实数 \(\displaystyle a\) 的取值范围为 \(\displaystyle \left( {-\infty , - 1\rbrack \cup \lbrack 4, + \infty }\right)\) .


\end{jiexi}
\end{problem}
\begin{problem} 
若关于 \(\displaystyle x\) 的不等式 \(\displaystyle \left| {x - 1}\right|  + \left| {{ax} - 1}\right|  \geq  {2x}\) 对于任意 \(\displaystyle x > 0\) 恒成立,则实数 \(\displaystyle a\) 的取值范围是\_\_\_\_\_.
\begin{jiexi}
【答案】 \(\displaystyle a \leq   - 1\) 或 \(\displaystyle a \geq  3\) ;

【解析】 \(\displaystyle \left| {x - 1}\right|  + \left| {{ax} - 1}\right|  \geq  {2x} \Leftrightarrow  \left| {\frac{1}{x} - 1}\right|  + \left| {\frac{1}{x} - a}\right|  \geq  2 \Leftrightarrow  \left| {t - 1}\right|  + \left| {t - a}\right|  \geq  2\left( {t > 0}\right)\) .

所以 \(\displaystyle {\left( \left| t - 1\right|  + \left| t - a\right| \right) }_{\min } = \left| {\left( {t - 1}\right)  - \left( {t - a}\right) }\right|  = \left| {a - 1}\right|  \geq  2\) ,解得 \(\displaystyle a \leq   - 1\) 或 \(\displaystyle a \geq  3\) ,故答案为: \(\displaystyle a \leq   - 1\) 或 \(\displaystyle a \geq  3\) ;


\end{jiexi}
\end{problem}
\begin{problem} 
若对任意 \(\displaystyle a \in  \left( {1, + \infty }\right)\) ,存在实数 \(\displaystyle x\) ,使得 \(\displaystyle \left| {x - a}\right|  - \left| {x + \frac{1}{a - 1}}\right|  + {x}^{2} + {a}^{2} \leq  {2ax} + a + m\) 成立,则实数 \(\displaystyle m\) 的最小值是\_\_\_\_\_.
\begin{jiexi}
【答案】 \(\displaystyle - 2 - 2\sqrt{2}\)

【详解】解: 因为对任意 \(\displaystyle a \in  \left( {1, + \infty }\right)\) ,存在实数 \(\displaystyle x\) ,使得 \(\displaystyle \left| {x - a}\right|  - \left| {x + \frac{1}{a - 1}}\right|  + {x}^{2} + {a}^{2} \leq  {2ax} + a + m\) 成

立,所以对任意 \(\displaystyle a \in  \left( {1, + \infty }\right)\) ,存在实数 \(\displaystyle x\) ,使得 \(\displaystyle \left| {x - a}\right|  - \left| {x + \frac{1}{a - 1}}\right|  + {\left( x - a\right) }^{2} \leq  a + m\) 成立,

因为 \(\displaystyle \left| {x - a}\right|  - \left| {x + \frac{1}{a - 1}}\right|  + {\left( x - a\right) }^{2} \geq   - \left| {a + \frac{1}{a - 1}}\right|\) ,当且仅当 \(\displaystyle x = a\) 时等号成立,

所以有 \(\displaystyle - \left| {a + \frac{1}{a - 1}}\right|  \leq  a + m\) 对任意 \(\displaystyle a \in  \left( {1, + \infty }\right)\) 恒成立,

即 \(\displaystyle - m \leq  {2a} + \frac{1}{a - 1} = 2\left( {a - 1}\right)  + \frac{1}{a - 1} + 2\) 对任意 \(\displaystyle a \in  \left( {1, + \infty }\right)\) 恒成立,

由于 \(\displaystyle 2\left( {a - 1}\right)  + \frac{1}{a - 1} + 2 \geq  2\sqrt{2} + 2\) ,当且仅当 \(\displaystyle 2\left( {a - 1}\right)  = \frac{1}{a - 1}\) ,即 \(\displaystyle a = \frac{\sqrt{2}}{2} + 1\) 时等号成立;

所以 \(\displaystyle - m \leq  2\sqrt{2} + 2\) ,即 \(\displaystyle m \geq   - 2\sqrt{2} - 2\) . 所以实数 \(\displaystyle m\) 的最小值是 \(\displaystyle - 2 - 2\sqrt{2}\)

\end{jiexi}
\end{problem}

\clearpage
\textbf{【课后作业】}\\
\textbf{一、填空题:本大题共12小题,每小题5分,共60分.把答案填在答题卡中的横线上}



\begin{hmwk} 
 (2010·湖北·一模)当 \(\displaystyle x > 1\) 时, \(\displaystyle x + \frac{4}{x - 1}\) 的最小值为\_\_\_\_\_.

\begin{jiexi}
【答案】 5

【难度】 0.65

【知识点】基本不等式求和的最小值

【分析】构造乘积为定值, 应用基本不等式求出最小值即可.

【详解】因为 \(\displaystyle x > 1\) ,

则 \(\displaystyle x + \frac{4}{x - 1} = x - 1 + \frac{4}{x - 1} + 1 \geq  2\sqrt{\left( {x - 1}\right)  \times  \left( \frac{4}{x - 1}\right) } + 1 = 2\sqrt{4} + 1 = 5\) ,

当 \(\displaystyle x - 1 = \frac{4}{x - 1},x = 3\) 时, \(\displaystyle x + \frac{4}{x - 1}\) 的最小值为 5 .

故答案为:5 .

\end{jiexi}
\end{hmwk}
\begin{hmwk} 
 (22-23 高一上·湖北武汉·期中) 若实数 \(\displaystyle x > 1,y > 1\) ,且 \(\displaystyle x + {2y} = 5\) ,则 \(\displaystyle \frac{1}{x - 1} + \frac{1}{y - 1}\) 的最小值为\_\_\_\_\_.

\begin{jiexi}
【答案】 \(\displaystyle \frac{3}{2} + \sqrt{2}\)

【难度】 0.85

【知识点】基本不等式“1”的妙用求最值

【分析】由已知变形可得出 \(\displaystyle \left( {x - 1}\right)  + 2\left( {y - 1}\right)  = 2\) ,将 \(\displaystyle \frac{1}{x - 1} + \frac{1}{y - 1}\) 与 \(\displaystyle \frac{1}{2}\left\lbrack  {\left( {x - 1}\right)  + 2\left( {y - 1}\right) }\right\rbrack\) 相乘,展开后利

用基本不等式可求得 \(\displaystyle \frac{1}{x - 1} + \frac{1}{y - 1}\) 的最小值.

【详解】因为实数 \(\displaystyle x > 1,y > 1\) ,且 \(\displaystyle x + {2y} = 5\) ,则 \(\displaystyle \left( {x - 1}\right)  + 2\left( {y - 1}\right)  = 2\) ,

所以, \(\displaystyle \frac{1}{x - 1} + \frac{1}{y - 1} = \frac{1}{2}\left\lbrack  {\left( {x - 1}\right)  + 2\left( {y - 1}\right) }\right\rbrack  \left( {\frac{1}{x - 1} + \frac{1}{y - 1}}\right)  = \frac{1}{2}\left\lbrack  {3 + \frac{x - 1}{y - 1} + \frac{2\left( {y - 1}\right) }{x - 1}}\right\rbrack\)

\(\displaystyle \geq  \frac{1}{2}\left\lbrack  {3 + 2\sqrt{\frac{x - 1}{y - 1} \cdot  \frac{2\left( {y - 1}\right) }{x - 1}}}\right\rbrack   = \frac{3}{2} + \sqrt{2},\)

当且仅当 \(\displaystyle \left\{  \begin{array}{l} x - 1 = \sqrt{2}\left( {y - 1}\right) \\  \left( {x - 1}\right)  + 2\left( {y - 1}\right)  = 2 \end{array}\right.\) 时,即当 \(\displaystyle \left\{  \begin{array}{l} x = 2\sqrt{2} - 1 \\  y = 3 - \sqrt{2} \end{array}\right.\) 时,等号成立.

因此, \(\displaystyle \frac{1}{x - 1} + \frac{1}{y - 1}\) 的最小值为 \(\displaystyle \frac{3}{2} + \sqrt{2}\) .

故答案为: \(\displaystyle \frac{3}{2} + \sqrt{2}\) .

\end{jiexi}
\end{hmwk}
\begin{hmwk} 
 (2023·山西大同·模拟预测) 已知 \(\displaystyle a > 0,b > 0,a \geq  \frac{1}{a} + \frac{2}{b},b \geq  \frac{1}{b} + \frac{2}{a}\) ,则 \(\displaystyle a + b\) 的最小值为\_\_\_\_\_.

\begin{jiexi}
【答案】 \(\displaystyle 2\sqrt{3}\)

【难度】 0.65

【知识点】由不等式的性质证明不等式、基本不等式“1”的妙用求最值

【分析】由已知可得 \(\displaystyle a + b \geq  \frac{3}{a} + \frac{3}{b}\) ,结合基本不等式求 \(\displaystyle {\left( a + b\right) }^{2}\) 的最小值,再求 \(\displaystyle a + b\) 的最小值.

【详解】因为 \(\displaystyle a \geq  \frac{1}{a} + \frac{2}{b},b \geq  \frac{1}{b} + \frac{2}{a}\) ,

所以 \(\displaystyle a + b \geq  \frac{3}{a} + \frac{3}{b}\) ,又 \(\displaystyle a > 0,b > 0\) ,

所以 \(\displaystyle {\left( a + b\right) }^{2} \geq  \left( {\frac{3}{a} + \frac{3}{b}}\right) \left( {a + b}\right)  = 6 + \frac{3b}{a} + \frac{3a}{b} \geq  {12}\) ,当且仅当 \(\displaystyle a = b = \sqrt{3}\) 时取等号.

所以 \(\displaystyle a + b \geq  2\sqrt{3}\) ,当且仅当 \(\displaystyle a = b = \sqrt{3}\) 时取等号.

所以 \(\displaystyle a + b\) 的最小值为 \(\displaystyle 2\sqrt{3}\) .

故答案为: \(\displaystyle 2\sqrt{3}\) .

\end{jiexi}
\end{hmwk}
\begin{hmwk} 
 (23-24 高一上·广东河源·阶段练习) 若正数 \(\displaystyle x,y\) 满足 \(\displaystyle \frac{1}{x} + \frac{8}{y} = 1\) ,则 \(\displaystyle x + {2y}\) 的最小值为\_\_\_\_\_.

\begin{jiexi}
【答案】 25

【难度】 0.65

【知识点】基本不等式“1”的妙用求最值

【分析】由题可得 \(\displaystyle x + {2y} = \left( {x + {2y}}\right) \left( {\frac{1}{x} + \frac{8}{y}}\right)\) ,化简利用基本不等式即可得出结论.

【详解】 \(\displaystyle \because\) 正数 \(\displaystyle x,y\) 满足 \(\displaystyle \frac{1}{x} + \frac{8}{y} = 1\) ,

\(\displaystyle \therefore x + {2y} = \left( {x + {2y}}\right) \left( {\frac{1}{x} + \frac{8}{y}}\right)  = 1 + {16} + \frac{2y}{x} + \frac{8x}{y} \geq  {17} + 2\sqrt{\frac{2y}{x} \cdot  \frac{8x}{y}} = {25}\) ,

当且仅当 \(\displaystyle \frac{2y}{x} = \frac{8x}{y}\) 即 \(\displaystyle y = {10},x = 5\) 时取等号.

故答案为: 25.

\end{jiexi}
\end{hmwk}
\begin{hmwk} 
 (23-24 高一下·河北·期末) 已知 \(\displaystyle a > 0,b > 0\) ,且 \(\displaystyle {9a} + b = {ab}\) ,则 \(\displaystyle a + {4b}\) 的最小值为\_\_\_\_\_.

\begin{jiexi}
【答案】 49 【难度】 0.65 【知识点】基本不等式求和的最小值、基本不等式“1”的妙用求最值

【分析】由 \(\displaystyle {9a} + b = {ab}\) 可得 \(\displaystyle \frac{1}{a} + \frac{9}{b} = 1\) ,即有 \(\displaystyle \left( {a + {4b}}\right) \left( {\frac{1}{a} + \frac{9}{b}}\right)  = {37} + \frac{4b}{a} + \frac{9a}{b}\) ,再由基本不等式可得最小值,

注意等号成立的条件.

【详解】因为 \(\displaystyle a > 0,b > 0\) 且 \(\displaystyle {9a} + b = {ab}\) ,所以 \(\displaystyle \frac{1}{a} + \frac{9}{b} = 1\) ,

所以 \(\displaystyle a + {4b} = \left( {a + {4b}}\right) \left( {\frac{1}{a} + \frac{9}{b}}\right)  = 1 + {36} + \frac{4b}{a} + \frac{9a}{b} \geq  {37} + 2\sqrt{\frac{4b}{a} \times  \frac{9a}{b}} = {37} + {12} = 4\) ,

当且仅当 \(\displaystyle \frac{4b}{a} = \frac{9a}{b}\) 即 \(\displaystyle a = 7,b = \frac{21}{2}\) 时取等号,

所以 \(\displaystyle a + {4b}\) 最小值为 49 .

故答案为: 49 .

\end{jiexi}
\end{hmwk}
\begin{hmwk} 
 (23-24 高一下·云南曲靖·阶段练习) 已知 \(\displaystyle x > 0\) , \(\displaystyle y > 0\) ,且 \(\displaystyle x + y = 3\) ,则 \(\displaystyle \frac{y}{x + 1} + \frac{1}{y}\) 的最小值为\_\_\_\_\_.

\begin{jiexi}
【答案】 \(\displaystyle \frac{5}{4}\)

【难度】 0.65

【知识点】基本不等式“1”的妙用求最值、基本不等式求和的最小值

【分析】根据分母特点,将 \(\displaystyle x + y = 3\) 化为 \(\displaystyle \left( {x + 1}\right)  + y = 4\) ,将 \(\displaystyle \frac{1}{y}\) 化为 \(\displaystyle \frac{4}{4y}\) . 然后用基本不等式即可.

【详解】由于 \(\displaystyle x + y = 3\) ,因此 \(\displaystyle \left( {x + 1}\right)  + y = 4\) ,

则 \(\displaystyle \frac{y}{x + 1} + \frac{1}{y} = \frac{y}{x + 1} + \frac{4}{4y} = \frac{y}{x + 1} + \frac{\left( {x + 1}\right)  + y}{4y} = \frac{y}{x + 1} + \frac{x + 1}{4y} + \frac{1}{4} \geq  2\sqrt{\frac{y}{x + 1} \cdot  \frac{x + 1}{4y}} + \frac{1}{4} = \frac{5}{4}\) ,

当且仅当 \(\displaystyle y = \frac{4}{3},x = \frac{5}{3}\) 时取等号.

故答案为: \(\displaystyle \frac{5}{4}\) .

\end{jiexi}
\end{hmwk}
\begin{hmwk} 
 (24-25 高一上一全国. 课前预习) 若不等式 \(\displaystyle \frac{x}{{x}^{2} + {3x} + 1} \leq  a\) 对一切正实数 \(\displaystyle x\) 都成立,则实数 \(\displaystyle a\) 的取值范围是\_\_\_\_\_.

\begin{jiexi}
【答案】 \(\displaystyle \left\lbrack  {\frac{1}{5}, + \infty }\right)\)

【难度】 0.65

【知识点】基本不等式求和的最小值、一元二次不等式在某区间上的恒成立问题

【分析】由题意 \(\displaystyle a \geq  \frac{x}{{x}^{2} + {3x} + 1}\left( {x > 0}\right)\) 恒成立,即 \(\displaystyle a \geq  {\left( \frac{x}{{x}^{2} + {3x} + 1}\right) }_{\max }\) ,然后由基本不等式求 \(\displaystyle \frac{x}{{x}^{2} + {3x} + 1}\) 的

最大值即可.

【详解】由题意 \(\displaystyle a \geq  \frac{x}{{x}^{2} + {3x} + 1}\) 恒成立,即 \(\displaystyle a \geq  {\left( \frac{x}{{x}^{2} + {3x} + 1}\right) }_{\max }\) ,

因为 \(\displaystyle x > 0\) ,所以 \(\displaystyle \frac{x}{{x}^{2} + {3x} + 1} = \frac{1}{x + \frac{1}{x} + 3} \leq  \frac{1}{2\sqrt{x \cdot  \frac{1}{x}} + 3} = \frac{1}{5}\) ,

当且仅当 \(\displaystyle x = \frac{1}{x}\) ,即 \(\displaystyle x = 1\) 时等号成立,

所以 \(\displaystyle \frac{x}{{x}^{2} + {3x} + 1}\) 的最大值为 \(\displaystyle \frac{1}{5}\) ,

所以 \(\displaystyle a \geq  \frac{1}{5}\) .

故答案为: \(\displaystyle \left\lbrack  {\frac{1}{5}, + \infty }\right)\) .

\end{jiexi}
\end{hmwk}
\begin{hmwk} 
 (23-24 高二下·浙江绍兴·期中) 已知 \(\displaystyle a > 0,b > 0\) ,且 \(\displaystyle a + {2b} = 1\) ,则 \(\displaystyle \frac{1}{b} + \frac{8}{a + b}\) 的最小值为\_\_\_\_\_.

\begin{jiexi}
【答案】 \(\displaystyle 9 + 4\sqrt{2}/4\sqrt{2} + 9\)

【难度】 0.65

【知识点】基本不等式“1”的妙用求最值

【分析】先变形: \(\displaystyle \frac{1}{b} + \frac{8}{a + b} = \left( {\frac{1}{b} + \frac{8}{a + b}}\right) \left\lbrack  {\left( {a + b}\right)  + b}\right\rbrack   = \frac{a + b}{b} + \frac{8b}{a + b} + 9\) ,再根据基本不等式求最值.

【详解】因为 \(\displaystyle a + {2b} = 1\) ,

所以 \(\displaystyle \frac{1}{b} + \frac{8}{a + b} = \left( {\frac{1}{b} + \frac{8}{a + b}}\right) \left\lbrack  {\left( {a + b}\right)  + b}\right\rbrack   = \frac{a + b}{b} + \frac{8b}{a + b} + 9\)

\(\displaystyle \geq  2\sqrt{\frac{a + b}{b} \cdot  \frac{8b}{a + b}} + 9 = 4\sqrt{2} + 9\)

当且仅当 \(\displaystyle \frac{a + b}{b} = \frac{8b}{a + b}\) ,即 \(\displaystyle b = \frac{2\sqrt{2} - 1}{7},a = \frac{9 - 4\sqrt{2}}{7}\) 取等号,

所以 \(\displaystyle \frac{1}{b} + \frac{8}{a + b}\) 的最小值为 \(\displaystyle 4\sqrt{2} + 9\) .

故答案为: \(\displaystyle 4\sqrt{2} + 9\) .

\end{jiexi}
\end{hmwk}
\begin{hmwk} 
 (23-24 高三下·江西·阶段练习) 设 \(\displaystyle a,b \in  {\mathrm{R}}_{ + }\) ,若 \(\displaystyle a + {4b} = 4\) ,则 \(\displaystyle \frac{\sqrt{a} + 2\sqrt{b}}{\sqrt{ab}}\) 的最小值为\_\_\_\_\_.

\begin{jiexi}
【答案】 \(\displaystyle 2\sqrt{2}\)

【难度】 0.65

【知识点】基本不等式求积的最大值 【分析】运用基本不等式求出 \(\displaystyle {ab}\) 的范围,再对 \(\displaystyle \frac{\sqrt{a} + 2\sqrt{b}}{\sqrt{ab}}\) 的分子运用基本不等式,放缩为 \(\displaystyle 2\sqrt{\frac{2}{\sqrt{ab}}}\) ,再根据等号成立条件, 运用不等式的传递性求解即可.

【详解】由 \(\displaystyle a,b \in  {\mathrm{R}}_{ + },a + {4b} = 4\) ,得 \(\displaystyle 4 = a + {4b} \geq  2\sqrt{a \cdot  {4b}} = 4\sqrt{ab}\) ,所以 \(\displaystyle \sqrt{ab} \leq  1\) ,

当且仅当 \(\displaystyle a = {4b} = 2\) 时取等号,

\(\displaystyle \frac{\sqrt{a} + 2\sqrt{b}}{\sqrt{ab}} \geq  \frac{2\sqrt{\sqrt{a} \cdot  2\sqrt{b}}}{\sqrt{ab}} = 2\sqrt{\frac{2\sqrt{ab}}{ab}} = 2\sqrt{\frac{2}{\sqrt{ab}}},\)

当且仅当 \(\displaystyle \sqrt{a} = 2\sqrt{b}\) 时取等号,

所以 \(\displaystyle \frac{\sqrt{a} + 2\sqrt{b}}{\sqrt{ab}} \geq  2\sqrt{\frac{2}{\sqrt{ab}}} \geq  2\sqrt{2}\) ,两个不等式等号成立条件相同,

所以 \(\displaystyle \frac{\sqrt{a} + 2\sqrt{b}}{\sqrt{ab}} \geq  2\sqrt{2}\) ,当且仅当 \(\displaystyle a = 2,b = \frac{1}{2}\) 时, \(\displaystyle \frac{\sqrt{a} + 2\sqrt{b}}{\sqrt{ab}}\) 取得最小值 \(\displaystyle 2\sqrt{2}\) .

故答案为: \(\displaystyle 2\sqrt{2}\) .

\end{jiexi}
\end{hmwk}
\begin{hmwk} 
 (23-24 高二下·天津滨海新·阶段练习) 若 \(\displaystyle x > 0,y >  - 2\) ,且 \(\displaystyle x + y = 1\) ,则 \(\displaystyle \frac{{x}^{2} + 1}{x} + \frac{{y}^{2}}{y + 2}\) 的最小值为\_\_\_\_\_. 
 
 \begin{jiexi}
【答案】 2 

【难度】 0.65 【知识点】基本不等式“1”的妙用求最值、条件等式求最值 

【分析】通分后利用已知化简,然后再变形为 \(\displaystyle \frac{4}{y + 2} + \frac{1}{x}\) ,利用常数代换,结合基本不等式可得. 

【详解】因为 \(\displaystyle x + y = 1\) , 所以 \(\displaystyle \frac{{x}^{2} + 1}{x} + \frac{{y}^{2}}{y + 2} = \frac{\left( {{x}^{2} + 1}\right) \left( {y + 2}\right)  + x{y}^{2}}{x\left( {y + 2}\right) } = \frac{{xy}\left( {x + y}\right)  + 2{x}^{2} + y + 2}{x\left( {y + 2}\right) }\)  \(\displaystyle = \frac{{xy} + 2{x}^{2} + y + 2}{x\left( {y + 2}\right) } = \frac{x\left( {y + {2x}}\right)  + y + 2}{x\left( {y + 2}\right) } = \frac{x\left( {x + 1}\right)  + y + 2}{x\left( {y + 2}\right) } = \frac{x + 1}{y + 2} + \frac{1}{x}\)  \(\displaystyle = \frac{2 - y}{y + 2} + \frac{1}{x} = \frac{4}{y + 2} + \frac{1}{x} - 1,\) 由于 \(\displaystyle x + y = 1\) ,所以 \(\displaystyle x + y + 2 = 3\) ,且 \(\displaystyle x > 0,y + 2 > 0\) ,

所以 \(\displaystyle \frac{4}{y + 2} + \frac{1}{x} = \frac{1}{3}\left( {\frac{4}{y + 2} + \frac{1}{x}}\right) \left( {x + y + 2}\right)  = \frac{1}{3}\left( {5 + \frac{4x}{y + 2} + \frac{y + 2}{x}}\right)  \geq  \frac{1}{3}\left( {5 + 2\sqrt{4}}\right)  = 3\) ,

所以 \(\displaystyle \frac{4}{y + 2} + \frac{1}{x} - 1 \geq  2\) ,当且仅当 \(\displaystyle \left\{  \begin{array}{l} \frac{4x}{y + 2} = \frac{y + 2}{x} \\  x + y = 1 \end{array}\right.\) ,即 \(\displaystyle x = 1,y = 0\) 时等号成立,

所以 \(\displaystyle \frac{{x}^{2} + 1}{x} + \frac{{y}^{2}}{y + 2}\) 的最小值为 2.

故答案为: 2

\end{jiexi}
\end{hmwk}
\begin{hmwk} 
 (2016 高二. 全国. 竞赛) 设 \(\displaystyle a > b > 0\) ,则 \(\displaystyle {a}^{2} + \frac{1}{ab} + \frac{1}{a\left( {a - b}\right) }\) 的最小值为\_\_\_\_\_.

\begin{jiexi}
【答案】 4

【难度】 0.65

【知识点】基本不等式求和的最小值

【分析】变形得 \(\displaystyle {a}^{2} + \frac{1}{ab} + \frac{1}{a\left( {a - b}\right) } = {a}^{2} - {ab} + \frac{1}{a\left( {a - b}\right) } + {ab} + \frac{1}{ab}\) ,再利用基本不等式即可求出结果.

【详解】因为 \(\displaystyle a > b > 0\) ,则 \(\displaystyle {a}^{2} > {ab}\) ,即 \(\displaystyle {a}^{2} - {ab} > 0\) ,

则 \(\displaystyle {a}^{2} + \frac{1}{ab} + \frac{1}{a\left( {a - b}\right) } = {a}^{2} - {ab} + \frac{1}{a\left( {a - b}\right) } + {ab} + \frac{1}{ab}\)

\(\displaystyle \geq  2\sqrt{\left( {{a}^{2} - {ab}}\right)  \times  \frac{1}{a\left( {a - b}\right) }} + 2\sqrt{{ab} \times  \frac{1}{ab}} = 4,\)

当且仅当 \(\displaystyle \frac{1}{a\left( {a - b}\right) } = a\left( {a - b}\right) ,{ab} = \frac{1}{ab}\) 时取等号,此时 \(\displaystyle a = \sqrt{2},b = \frac{\sqrt{2}}{2}\) ,

故答案为: 4 .

\end{jiexi}
\end{hmwk}
\begin{hmwk} 
 (23-24 高一上· 山东菏泽·阶段练习) 若两个正实数 \(\displaystyle x,y\) 满足 \(\displaystyle x + y = 3\) ,且不等式 \(\displaystyle \frac{4}{x + 1} + \frac{16}{y} > m\) 恒成立,则实数 \(\displaystyle m\) 的取值范围为\_\_\_\_\_.

\begin{jiexi}
【答案】 \(\displaystyle \left( {-\infty ,9}\right)\)

【难度】 0.65

【知识点】基本不等式求和的最小值、基本不等式“1”的妙用求最值、基本不等式的恒成立问题

【分析】根据等式变形,利用常值代换法凑项,运用基本不等式求得 \(\displaystyle {\left( \frac{4}{x + 1} + \frac{16}{y}\right) }_{\min }\) 即得.

【详解】因为两个正实数 \(\displaystyle x,y\) 满足 \(\displaystyle x + y = 3\) ,则 \(\displaystyle \left( {x + 1}\right)  + y = 4\) ,

故 \(\displaystyle \frac{4}{x + 1} + \frac{16}{y} = \frac{1}{4}\left( {\frac{4}{x + 1} + \frac{16}{y}}\right) \left\lbrack  {\left( {x + 1}\right)  + y}\right\rbrack   = \frac{y}{x + 1} + \frac{4\left( {x + 1}\right) }{y} + 5\)

\(\displaystyle \geq  2\sqrt{\frac{y}{x + 1} \cdot  \frac{4\left( {x + 1}\right) }{y}} + 5 = 9\) ,当且仅当 \(\displaystyle x = \frac{1}{3},y = \frac{8}{3}\) 时取等号,

因不等式 \(\displaystyle \frac{4}{x + 1} + \frac{16}{y} > m\) 恒成立,则 \(\displaystyle m < {\left( \frac{4}{x + 1} + \frac{16}{y}\right) }_{\min }\) ,故 \(\displaystyle m < 9\) .

故答案为: \(\displaystyle \left( {-\infty ,9}\right)\) .



\end{jiexi}
\end{hmwk}

\textbf{二、单选题:本大题共 4 小题,每小题 5 分,共 20 分. 把答案填在答题卡中的横线上.}
\begin{hmwk} 
 (23-24 高一上·广东深圳·期中) \(\displaystyle {x}^{2} + \frac{10}{{x}^{2} + 1}\) 的最小值为(   )

A. \(\displaystyle 2\sqrt{10} - 1\) B. \(\displaystyle 2\sqrt{10}\) C. \(\displaystyle 2\sqrt{5} - 1\) D. 10

\begin{jiexi}
【答案】A

【难度】 0.85

【知识点】基本不等式求和的最小值

【分析】由题意可得 \(\displaystyle {x}^{2} + \frac{10}{{x}^{2} + 1} = {x}^{2} + 1 + \frac{10}{{x}^{2} + 1} - 1\) ,再由基本不等式求解即可.

【详解】 \(\displaystyle {x}^{2} + \frac{10}{{x}^{2} + 1} = {x}^{2} + 1 + \frac{10}{{x}^{2} + 1} - 1 \geq  2\sqrt{\left( {{x}^{2} + 1}\right)  \cdot  \frac{10}{{x}^{2} + 1}} - 1 = 2\sqrt{10} - 1\) ,

当且仅当 \(\displaystyle {x}^{2} + 1 = \frac{10}{{x}^{2} + 1}\) ,即 \(\displaystyle x =  \pm  \sqrt{\sqrt{10} - 1}\) 时,等号成立.

所以 \(\displaystyle {x}^{2} + \frac{10}{{x}^{2} + 1}\) 的最小值为 \(\displaystyle 2\sqrt{10} - 1\) .

故选: A.

\end{jiexi}
\end{hmwk}
\begin{hmwk} 
 (24-25 高一上·全国·课后作业) 若 \(\displaystyle 0 < x < 4\) ,则 \(\displaystyle \sqrt{{2x}\left( {4 - x}\right) }\) 有(   )

A. 最小值 0 B. 最大值 2

C. 最大值 \(\displaystyle 2\sqrt{2}\) D. 不能确定

\begin{jiexi}
【答案】 \(\displaystyle \mathrm{C}\)

【难度】 0.65

【知识点】基本不等式求积的最大值

【分析】根据基本不等式求乘积的最大值, 再检验最小值的情况即可得解.

【详解】由基本不等式,得 \(\displaystyle \sqrt{{2x}\left( {4 - x}\right) } = \sqrt{2} \cdot  \sqrt{x\left( {4 - x}\right) } \leq  \sqrt{2} \cdot  \frac{x + \left( {4 - x}\right) }{2} = 2\sqrt{2}\) ,

当且仅当 \(\displaystyle x = 4 - x\) ,即 \(\displaystyle x = 2\) 时等号成立,

故 \(\displaystyle \sqrt{{2x}\left( {4 - x}\right) }\) 有最大值 \(\displaystyle 2\sqrt{2}\) ,故 \(\displaystyle \mathrm{C}\) 正确, \(\displaystyle \mathrm{{BD}}\) 错误;

令 \(\displaystyle \sqrt{{2x}\left( {4 - x}\right) } = 0\) ,解得 \(\displaystyle x = 0\) 或 \(\displaystyle x = 4\) , 又 \(\displaystyle 0 < x < 4\) ,所以 \(\displaystyle \sqrt{{2x}\left( {4 - x}\right) }\) 取不到函数值 0,故 \(\displaystyle \mathrm{A}\) 错误. 故选: C. 

\end{jiexi}
\end{hmwk}
\begin{hmwk} 
 (2024·河南信阳·模拟预测) \(\displaystyle a > 0,b > 0,\frac{1}{a} + \frac{2}{b} = 1\) ,则 \(\displaystyle \frac{1}{a - 1} + \frac{3}{b - 2}\) 的最小值为 (   ) 
 
 A. \(\displaystyle \sqrt{3}\) B. \(\displaystyle 2\sqrt{3}\) C. \(\displaystyle \sqrt{6}\) D. 6 
 
 \begin{jiexi}
【答案】 \(\displaystyle \mathrm{C}\) 

【难度】 0.65 【知识点】条件等式求最值、基本不等式求和的最小值 【分析】由已知可得 \(\displaystyle \left( {a - 1}\right) \left( {b - 2}\right)  = 2\) ,利用基本不等式求 \(\displaystyle \frac{1}{a - 1} + \frac{3}{b - 2}\) 的最小值. 【详解】 \(\displaystyle a > 0,b > 0,\frac{1}{a} + \frac{2}{b} = 1\) ,则 \(\displaystyle {ab} = {2a} + b\) ,且 \(\displaystyle a > 1,b > 2\) , 整理得到 \(\displaystyle \left( {a - 1}\right) \left( {b - 2}\right)  = 2\) , 所以 \(\displaystyle \frac{1}{a - 1} + \frac{3}{b - 2} \geq  2\sqrt{\frac{3}{\left( {a - 1}\right) \left( {b - 2}\right) }} = \sqrt{6}\) ,当且仅当 \(\displaystyle \frac{1}{a - 1} = \frac{3}{b - 2}\) ,即 \(\displaystyle a = \frac{\sqrt{6}}{3} + 1,b = \sqrt{6} + 2\) 时取等号. 即 \(\displaystyle \frac{1}{a - 1} + \frac{3}{b - 2}\) 的最小值为 \(\displaystyle \sqrt{6}\) .

故选: C.

\end{jiexi}
\end{hmwk}
\begin{hmwk} 
 (20-21 高三下·湖南·阶段练习) 数学里有一种证明方法叫做 Proofswithoutwords, 也称之为无字证明,一般是指仅用图象语言而无需文字解释就能不证自明的数学命题,由于这种证明方法的特殊性, 无字证明被认为比严格的数学证明更为优雅. 现有如图所示图形,在等腰直角三角形 \(\displaystyle \mathrm{V}{ABC}\) 中,点 \(\displaystyle O\) 为斜边 \(\displaystyle {AB}\) 的中点,点 \(\displaystyle D\) 为斜边 \(\displaystyle {AB}\) 上异于顶点的一个动点,设 \(\displaystyle {AD} = a,{BD} = b\) ,则该图形可以完成的无字证明为(   )


\includegraphics[max width=0.3\textwidth]{images/01968fa5-d2e3-7182-940c-34b4aef08981_26_183_1568_396_253_0.jpg}

 

A. \(\displaystyle \frac{a + b}{2} \geq  \sqrt{ab}\left( {a > 0,b > 0}\right)\) B. \(\displaystyle \frac{a + b}{2} \leq  \sqrt{\frac{{a}^{2} + {b}^{2}}{2}}\left( {a > 0,b > 0}\right)\)

C. \(\displaystyle \frac{2ab}{a + b} \leq  \sqrt{ab}\left( {a > 0,b > 0}\right)\) D. \(\displaystyle {a}^{2} + {b}^{2} \geq  2\sqrt{ab}\left( {a > 0,b > 0}\right)\)

\begin{jiexi}
【答案】B 【难度】 0.85 【知识点】不等式、由基本不等式证明不等关系 【分析】通过图形,并因为 \(\displaystyle {AD} = a,{BD} = b\) ,所以 \(\displaystyle {OC} = \frac{a + b}{2},{OD} = \left| \frac{a - b}{2}\right|\) ,从而可以通过勾股定理求得 \(\displaystyle {CD}\) ,又因为 \(\displaystyle {CD} \geq  {OC}\) ,从而可以得到答案.

【详解】 \(\displaystyle \because \mathrm{V}{ABC}\) 等腰直角三角形, \(\displaystyle O\) 为斜边 \(\displaystyle {AB}\) 的中点, \(\displaystyle {AD} = a,{BD} = b\)

\(\displaystyle \therefore {OC} = \frac{a + b}{2},\;{OD} = \left| \frac{a - b}{2}\right|\)

\(\displaystyle \because {OC} \bot  {AB}\)

\(\displaystyle \therefore C{D}^{2} = O{C}^{2} + O{D}^{2} = {\left( \frac{a + b}{2}\right) }^{2} + {\left( \frac{a - b}{2}\right) }^{2} = \frac{{a}^{2} + {b}^{2}}{2}\)

\(\displaystyle \therefore {CD} = \sqrt{\frac{{a}^{2} + {b}^{2}}{2}}\)

而 \(\displaystyle {CD} \geq  {OC}\) ,所以 \(\displaystyle \sqrt{\frac{{a}^{2} + {b}^{2}}{2}} \geq  \frac{a + b}{2}\left( {a > 0,b > 0}\right)\) ,故选项 \(\displaystyle \mathrm{B}\) 正确.

故选:B


\end{jiexi}
\end{hmwk}
\textbf{三、解答题:本大题共 6 小题,共 70 分.解答应写出必要的文字说明、证明过程或演算步骤.}
\begin{hmwk} 
 (23-24 高一上·广东深圳·期中) 已知 \(\displaystyle {a}^{2} + 8{b}^{2} = 4\) .

(1)若 \(\displaystyle a\) 与 \(\displaystyle b\) 均为正数,求 \(\displaystyle {ab}\) 的最大值;

(2)若 \(\displaystyle a\) 与 \(\displaystyle b\) 均为负数,求 \(\displaystyle \frac{1}{{a}^{2}} + \frac{2}{{b}^{2}}\) 的最小值.

\begin{jiexi}[65]
【答案】(1) \(\displaystyle \frac{\sqrt{2}}{2}\)

(2) \(\displaystyle \frac{25}{4}\)

【难度】 0.65

【知识点】基本不等式求积的最大值、基本不等式“1”的妙用求最值

【分析】(1)根据基本不等式和为定值求解乘积的最值即可;

(2)利用基本不等式“1”的巧用求解最值即可.

【详解】(1)因为 \(\displaystyle a\) 与 \(\displaystyle b\) 均为正数,所以 \(\displaystyle {a}^{2} + 8{b}^{2} = 4 \geq  2\sqrt{{a}^{2} \cdot  8{b}^{2}} = 4\sqrt{2}{ab}\) ,

当且仅当 \(\displaystyle {a}^{2} = 8{b}^{2}\) ,即 \(\displaystyle a = 2\sqrt{2}b = \sqrt{2}\) 时,等号成立,

所以 \(\displaystyle {ab} \leq  \frac{\sqrt{2}}{2}\) ,所以 \(\displaystyle {ab}\) 的最大值为 \(\displaystyle \frac{\sqrt{2}}{2}\) .

(2)因为 \(\displaystyle a\) 与 \(\displaystyle b\) 均为负数,所以 \(\displaystyle {a}^{2} > 0,{b}^{2} > 0\) ,

所以 \(\displaystyle \frac{1}{{a}^{2}} + \frac{2}{{b}^{2}} = \frac{1}{4}\left( {{a}^{2} + 8{b}^{2}}\right) \left( {\frac{1}{{a}^{2}} + \frac{2}{{b}^{2}}}\right)  = \frac{1}{4}\left( {{17} + \frac{8{b}^{2}}{{a}^{2}} + \frac{2{a}^{2}}{{b}^{2}}}\right)  \geq  \frac{1}{4}\left( {{17} + 2\sqrt{16}}\right)  = \frac{25}{4}\) ,

当且仅当 \(\displaystyle \frac{8{b}^{2}}{{a}^{2}} = \frac{2{a}^{2}}{{b}^{2}}\) ,即 \(\displaystyle a = \sqrt{2}b =  - \frac{2\sqrt{5}}{5}\) 时,等号成立,

所以 \(\displaystyle \frac{1}{{a}^{2}} + \frac{2}{{b}^{2}}\) 的最小值为 \(\displaystyle \frac{25}{4}\) .

\end{jiexi}
\end{hmwk}
\sclear
\begin{hmwk} 
 (23-24 高一上·山东聊城·阶段练习) (1) 已知 \(\displaystyle {12} < a < {60},{15} < b < {36}\) , 求 \(\displaystyle a - {2b}\) 的取值范围.

(2)已知 \(\displaystyle x > 0,y > 0\) 且 \(\displaystyle \frac{1}{x} + \frac{9}{y} = 1\) ,求使不等式 \(\displaystyle x + y \geq  m\) 恒成立的实数 \(\displaystyle m\) 的取值范围.

\begin{jiexi}[65]
【答案】(1) \(\displaystyle - {60} < a - {2b} < {30}\) ;(2) \(\displaystyle m \leq  {16}\)

【难度】 0.65

【知识点】利用不等式求值或取值范围、基本不等式“1”的妙用求最值、基本不等式的恒成立问题

【分析】(1)根据不等式的性质通过乘积及和的运算得出式子范围即可;

(2)通过基本不等式 1 的活用得出最小值即可转化恒成立问题求参.

【详解】(1)因为 \(\displaystyle {15} < b < {36}\) ,所以 \(\displaystyle - {72} <  - {2b} <  - {30}\) .

又 \(\displaystyle {12} < a < {60}\) ,所以 \(\displaystyle {12} - {72} < a - {2b} < {60} - {30}\) ,即 \(\displaystyle - {60} < a - {2b} < {30}\) .

(2) 由 \(\displaystyle \frac{1}{x} + \frac{9}{y} = 1\) ,

则 \(\displaystyle x + y = \left( {x + y}\right) \left( {\frac{1}{x} + \frac{9}{y}}\right)  = {10} + \frac{9x}{y} + \frac{y}{x} \geq  {10} + 2\sqrt{\frac{9x}{y}\frac{y}{x}} = {16}\) .

当且仅当 \(\displaystyle \left\{  \begin{array}{l} x + y = {16} \\  \frac{9x}{y} = \frac{y}{x} \end{array}\right.\) 即 \(\displaystyle \left\{  \begin{array}{l} x = 4 \\  y = {12} \end{array}\right.\) 时取到最小值 16.

若 \(\displaystyle x + y \geq  m\) 恒成立,则 \(\displaystyle m \leq  {16}\) .

\end{jiexi}
\end{hmwk}
\begin{hmwk} 
 (23-24 高一上·江苏·阶段练习) (1) 若 \(\displaystyle x < 0\) ,求 \(\displaystyle y = \frac{12}{x} + {3x}\) 的最大值.

( 2 )已知 \(\displaystyle 0 < x < \frac{1}{3}\) ,求 \(\displaystyle y = x\left( {1 - {3x}}\right)\) 的最大值.

\begin{jiexi}[55]
【答案】(1)-12(2) \(\displaystyle \frac{1}{12}\)

【难度】 0.65

【知识点】基本不等式求积的最大值、基本不等式求和的最小值

【分析】( 1 )由已知结合基本不等式求出 \(\displaystyle - y =  - \frac{12}{x} + \left( {-{3x}}\right)\) 的最小值即可得 \(\displaystyle y = \frac{12}{x} + {3x}\) 的最大值.

( 2 )先用配凑法将 \(\displaystyle y = x\left( {1 - {3x}}\right)\) 变形为 \(\displaystyle y = \frac{1}{3} \times  {3x}\left( {1 - {3x}}\right)\) ,再利用基本不等式即可求解.

【详解】(1)因为 \(\displaystyle x < 0\) ,所以 \(\displaystyle - x > 0\) ,

所以 \(\displaystyle - y =  - \frac{12}{x} + \left( {-{3x}}\right)  \geq  2\sqrt{-\frac{12}{x} \times  \left( {-{3x}}\right) } = 2 \times  6 = {12}\) ,

当且仅当 \(\displaystyle - \frac{12}{x} =  - {3x}\) 即 \(\displaystyle x =  - 2\) 时取等号,

所以 \(\displaystyle y = \frac{12}{x} + {3x} \leq   - {12}\) 即 \(\displaystyle y = \frac{12}{x} + {3x}\) 最大值为 -12 .

(2)因为 \(\displaystyle 0 < x < \frac{1}{3}\) ,

所以 \(\displaystyle 0 < {3x} < 1\) ,则 \(\displaystyle 1 - {3x} > 0\) ,

所以 \(\displaystyle y = x\left( {1 - {3x}}\right)  = \frac{1}{3} \times  {3x}\left( {1 - {3x}}\right)  \leq  \frac{1}{3}{\left\lbrack  \frac{{3x} + \left( {1 - {3x}}\right) }{2}\right\rbrack  }^{2} = \frac{1}{12}\) ,

当且仅当 \(\displaystyle {3x} = 1 - {3x}\) 时,即 \(\displaystyle x = \frac{1}{6}\) 时取等号,

所以 \(\displaystyle y = x\left( {1 - {3x}}\right)\) 的最大值是 \(\displaystyle \frac{1}{12}\) .

\end{jiexi}
\end{hmwk}
\begin{hmwk} 
 (23-24 高一上·江苏南京·期中) (1) 设 \(\displaystyle a,b,c,d\) 为实数,求证: \(\displaystyle {ab} + {bc} + {cd} + {ad} \leq  {a}^{2} + {b}^{2} + {c}^{2} + {d}^{2}\) ;

(2)已知 \(\displaystyle a,b \in  \mathbf{R}\) ,求证: \(\displaystyle \frac{{6}^{a}}{{36}^{a + 1} + 1} \leq  \frac{5}{6} - b + \frac{{b}^{2}}{3}\) .

\begin{jiexi}[45]
【答案】(1)证明见解析;(2)证明见解析

【难度】 0.65

【知识点】由基本不等式证明不等关系、作差法比较代数式的大小

【分析】(1)利用作差法化简证明即可;

(2)利用基本不等式结合配方法证明即可.

【详解】( 1 )因为 \(\displaystyle 2\left( {{a}^{2} + {b}^{2} + {c}^{2} + {d}^{2}}\right)  - 2\left( {{ab} + {bc} + {cd} + {ad}}\right)\)

\(\displaystyle = {\left( a - b\right) }^{2} + {\left( b - c\right) }^{2} + {\left( c - d\right) }^{2} + {\left( a - d\right) }^{2} \geq  0\) ,

当且仅当 \(\displaystyle a = b = c = d\) 时,等号成立,

所以 \(\displaystyle 2\left( {{a}^{2} + {b}^{2} + {c}^{2} + {d}^{2}}\right)  \geq  2\left( {{ab} + {bc} + {cd} + {ad}}\right)\) ,

所以 \(\displaystyle {ab} + {bc} + {cd} + {ad} \leq  {a}^{2} + {b}^{2} + {c}^{2} + {d}^{2}\) ;

( 2 )因为 \(\displaystyle {6}^{a + 2} + \frac{1}{{6}^{a}} \geq  2\sqrt{{6}^{a + 2} \cdot  \frac{1}{{6}^{a}}} = {12}\) ,当且仅当 \(\displaystyle {6}^{a + 2} = \frac{1}{{6}^{a}}\) ,即 \(\displaystyle a =  - 1\) 时取等号,

所以 \(\displaystyle \frac{{6}^{a}}{{36}^{a + 1} + 1} = \frac{1}{{6}^{a + 2} + \frac{1}{{6}^{a}}} \leq  \frac{1}{12}\) ,当且仅当 \(\displaystyle {6}^{a + 2} = \frac{1}{{6}^{a}}\) ,即 \(\displaystyle a =  - 1\) 时取等号,

因为 \(\displaystyle \frac{5}{6} - b + \frac{{b}^{2}}{3} = \frac{1}{3}{\left( b - \frac{3}{2}\right) }^{2} + \frac{1}{12} \geq  \frac{1}{12}\) ,

综上 \(\displaystyle \frac{{6}^{a}}{{36}^{a + 1} + 1} \leq  \frac{5}{6} - b + \frac{{b}^{2}}{3}\) .

\end{jiexi}
\end{hmwk}
\sclear
\begin{hmwk} 
 (24-25 高一上·上海·课后作业) (1) 已知 \(\displaystyle x\) 、 \(\displaystyle y\) 都是正数,求证: \(\displaystyle \left( {x + y}\right) \left( {{x}^{2} + {y}^{2}}\right) \left( {{x}^{3} + {y}^{3}}\right)  \geq  8{x}^{3}{y}^{3}\) ;

(2)已知 \(\displaystyle a > 0\) , \(\displaystyle b > 0\) , \(\displaystyle c > 0\) ,求证: \(\displaystyle \frac{bc}{a} + \frac{ac}{b} + \frac{ab}{c} \geq  a + b + c\) .

\begin{jiexi}
【答案】(1)证明见解析;(2)证明见解析

【难度】 0.65

【知识点】由基本不等式证明不等关系

【分析】( 1 )对 \(\displaystyle x + y,{x}^{2} + {y}^{2},{x}^{3} + {y}^{3}\) 分别利用基本不等式,然后将得到的式子相乘可得结论;

( 2 )对 \(\displaystyle \frac{bc}{a} + \frac{ac}{b},\frac{bc}{a} + \frac{ab}{c},\frac{ac}{b} + \frac{ab}{c}\) 分别利用基本不等式,然后将得到的式子相加化简可得结论.

【详解】证明:(1) \(\displaystyle \because x\) 、 \(\displaystyle y\) 都是正数,

\(\displaystyle \therefore x + y \geq  2\sqrt{xy} > 0,\;{x}^{2} + {y}^{2} \geq  2\sqrt{{x}^{2}{y}^{2}} > 0,\;{x}^{3} + {y}^{3} \geq  2\sqrt{{x}^{3}{y}^{3}} > 0\) ,

\(\displaystyle \therefore \left( {x + y}\right) \left( {{x}^{2} + {y}^{2}}\right) \left( {{x}^{3} + {y}^{3}}\right)  \geq  2\sqrt{xy} \cdot  2\sqrt{{x}^{2}{y}^{2}} \cdot  2\sqrt{{x}^{3}{y}^{3}} = 8{x}^{3}{y}^{3}\) ,

当且仅当 \(\displaystyle x = y\) 时,等号成立.

(2) \(\displaystyle \because a > 0,b > 0,c > 0\) ,

\(\displaystyle \therefore \frac{bc}{a} + \frac{ac}{b} \geq  {2c},\frac{bc}{a} + \frac{ab}{c} \geq  {2b},\frac{ac}{b} + \frac{ab}{c} \geq  {2a}\) ,

\(\displaystyle \therefore 2\left( {\frac{bc}{a} + \frac{ac}{b} + \frac{ab}{c}}\right)  \geq  2\left( {a + b + c}\right)\) ,

故 \(\displaystyle \frac{bc}{a} + \frac{ac}{b} + \frac{ab}{c} \geq  a + b + c\) ,当且仅当 \(\displaystyle \frac{bc}{a} = \frac{ac}{b} = \frac{ab}{c}\) ,

即 \(\displaystyle a = b = c\) 时等号成立.
\end{jiexi}
\end{hmwk}
\end{document}
