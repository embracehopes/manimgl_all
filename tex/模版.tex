\documentclass[12pt]{article}

% --- 包设置 ---
\usepackage{amsmath}         % 用于高级数学公式
\usepackage{amsfonts}        % 用于数学字体
\usepackage{amssymb}         % 用于数学符号
\usepackage{ctex}            % 中文支持包
\usepackage{geometry}        % 设置页面边距
\geometry{a4paper, left=2.5cm, right=2.5cm, top=2.5cm, bottom=2.5cm}

% --- 文档信息 ---
\title{物理题解析 (第19, 20题)}
\author{胡华烁}
\date{\today}

% --- 文档开始 ---
\begin{document}

\maketitle
\tableofcontents
\newpage

\section*{第六大题、带电粒子在电场中的运动}

\subsection*{19. (共7分)}

\subsubsection*{(1) (2分) 试证明电场强度的单位 1 N/C = 1 V/m。}
\textbf{证明:} \\
根据电势差的定义,在匀强电场中,沿电场线方向两点间的电势差 $U$、电场强度 $E$ 和距离 $d$ 的关系为:
\[ U = E \cdot d \]
由此可得,电场强度的单位可表示为 \textbf{伏特/米 (V/m)}。

\vspace{1em}
另外,根据电场强度的定义,电场中某点的场强 $E$ 等于置于该点的电荷 $q$ 所受的电场力 $F$ 与该电荷电量的比值:
\[ E = \frac{F}{q} \]
由此可得,电场强度的单位也可表示为 \textbf{牛顿/库仑 (N/C)}。

\vspace{1em}
我们可以从基本单位的定义来推导证明这两个单位是等价的:
\begin{itemize}
    \item 1 伏特 (V) 的定义是 "移动1库仑 (C) 电荷做1焦耳 (J) 的功",即 $1 \text{ V} = 1 \frac{\text{J}}{\text{C}}$。
    \item 1 焦耳 (J) 的定义是 "用1牛顿 (N) 的力使物体在力的方向上移动1米 (m)",即 $1 \text{ J} = 1 \text{ N} \cdot \text{m}$。
\end{itemize}
将以上定义代入 V/m 中进行推导:
\[ 1 \frac{\text{V}}{\text{m}} = \frac{1 \frac{\text{J}}{\text{C}}}{1 \text{m}} = \frac{1 \frac{\text{N} \cdot \text{m}}{\text{C}}}{1 \text{m}} = 1 \frac{\text{N}}{\text{C}} \]
\hfill \textbf{证毕。}

\subsubsection*{(2) (2分) 电场强度和电势分别是}
\begin{itemize}
    \item \textbf{电场强度}:用来描述电场的力的性质,它既有大小,也有方向(规定为正电荷在该点所受电场力的方向)。因此,电场强度是\textbf{矢量}。
    \item \textbf{电势}:用来描述电场的能的性质,是电场中某一点相对零电势点的电势差,只有大小,没有方向。因此,电势是\textbf{标量}。
\end{itemize}
\textbf{答案: C. 矢量 标量}

\subsubsection*{(3) (3分) 该平行板电容器,带电荷量为Q时两极板电势差为U,若带电荷量变为2Q,则}
此题的隐含条件是电容器充电后与电源断开,成为一个孤立的带电系统。
\begin{itemize}
    \item \textbf{电容 (C)}:平行板电容器的电容由其自身的物理结构决定,与电荷量 $Q$ 和电势差 $U$ 无关。只要电容器的物理结构不变,其电容 $C$ 就保持不变。
    \item \textbf{电荷量 (Q) 与电势差 (U) 的关系}:三者满足关系式 $Q = CU$,可推得 $U = \frac{Q}{C}$。
\end{itemize}
当电荷量变为 $Q' = 2Q$ 时,由于电容 $C$ 不变,新的电势差 $U'$ 为:
\[ U' = \frac{Q'}{C} = \frac{2Q}{C} = 2 \left(\frac{Q}{C}\right) = 2U \]
\textbf{答案: D. 电容不变,两极板电势差变为原来的2倍}

\subsection*{20. (共10分)}

\subsubsection*{(1) (2分) 该粒子在电场中所做的运动}
\begin{itemize}
    \item \textbf{受力分析}:粒子在电场中仅受到恒定的电场力 $\vec{F} = q\vec{E}$,该力的方向与粒子的初速度 $v_0$ 方向垂直。
    \item \textbf{运动分析}:由于粒子受到一个大小和方向都不变的恒定合外力,且该力与初速度方向垂直,其运动轨迹为一条抛物线。这种运动是\textbf{匀加速曲线运动}。
\end{itemize}
\textbf{答案: B. 匀加速曲线运动}

\subsubsection*{(2) (计算,4分) 该粒子在电场中运动的时间}
将运动分解为水平(x轴)和竖直(y轴)方向。
\begin{itemize}
    \item 水平方向:匀速直线运动,$v_x = v_0$。
    \item 竖直方向:初速度为0的匀加速直线运动,加速度为 $a_y = \frac{qE}{m}$。在时刻 $t$ 的竖直分速度为 $v_y = a_y t = \frac{qE}{m}t$。
\end{itemize}
粒子射出时,速度方向与初速度方向成 $30^\circ$ 角。由速度合成的几何关系可知:
\begin{align*}
\tan(30^\circ) &= \frac{v_y}{v_x} = \frac{\frac{qE}{m}t}{v_0} \\
\frac{1}{\sqrt{3}} &= \frac{qEt}{mv_0} \\
\implies t &= \frac{mv_0}{\sqrt{3}qE}
\end{align*}
\textbf{答案:} 粒子在电场中运动的时间为 $\displaystyle \frac{mv_0}{\sqrt{3}qE}$。

\subsubsection*{(3) (计算,4分) 粒子在这一过程中电势能的变化量}
电势能的变化量 $\Delta U_e$ 等于电场力做功 $W_e$ 的负值,而电场力做的功又等于粒子动能的变化量 $\Delta K$。所以 $\Delta U_e = -W_e = -\Delta K$。
\begin{itemize}
    \item \textbf{初动能 $K_i$}:
    \[ K_i = \frac{1}{2}mv_0^2 \]
    \item \textbf{末动能 $K_f$}:末速度的大小 $v_f$ 可由分速度求得:
    \[ v_y = v_x \tan(30^\circ) = \frac{v_0}{\sqrt{3}} \]
    \[ v_f^2 = v_x^2 + v_y^2 = v_0^2 + \left(\frac{v_0}{\sqrt{3}}\right)^2 = v_0^2 + \frac{1}{3}v_0^2 = \frac{4}{3}v_0^2 \]
    因此末动能为:
    \[ K_f = \frac{1}{2}mv_f^2 = \frac{1}{2}m\left(\frac{4}{3}v_0^2\right) = \frac{2}{3}mv_0^2 \]
    \item \textbf{动能变化量 $\Delta K$}:
    \[ \Delta K = K_f - K_i = \frac{2}{3}mv_0^2 - \frac{1}{2}mv_0^2 = \left(\frac{4}{6} - \frac{3}{6}\right)mv_0^2 = \frac{1}{6}mv_0^2 \]
    \item \textbf{电势能变化量 $\Delta U_e$}:
    \[ \Delta U_e = -\Delta K = -\frac{1}{6}mv_0^2 \]
\end{itemize}
\textbf{答案:} 粒子在此过程中电势能的变化量为 $\displaystyle -\frac{1}{6}mv_0^2$。

\end{document}