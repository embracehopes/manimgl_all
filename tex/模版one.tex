\documentclass{ctexart}

\usepackage{amsmath}
\usepackage{amssymb}
\usepackage[svgnames]{xcolor}
\usepackage[most]{tcolorbox}
\usepackage{fancyhdr}
\usepackage{geometry}

% 设置页面边距
\geometry{a4paper, left=2.5cm, right=2.5cm, top=2.5cm, bottom=2.5cm}

% --- 自定义定理和提示框样式 ---

% 紫色定理框样式 (定理 1.1, 1.2)
\newtcolorbox{purpletheorem}[1]{
    enhanced,
    colback=purple!5!white,      % 背景颜色
    colframe=purple!75!black,    % 边框颜色
    fonttitle=\bfseries,
    attach boxed title to top left={yshift=-2mm, xshift=3mm},
    boxed title style={
        colback=purple!75!black, % 标题框背景颜色
        sharp corners,
        top=2pt,
        bottom=2pt,
    },
    title=\textbf{定理 #1}
}

% 蓝色定理框样式 (定理 1.3)
\newtcolorbox{bluetheorem}[1]{
    enhanced,
    colback=cyan!10!white,       % 背景颜色
    colframe=blue!75!black,      % 边框颜色
    fonttitle=\bfseries,
    attach boxed title to top left={yshift=-2mm, xshift=3mm},
    boxed title style={
        colback=blue!75!black,   % 标题框背景颜色
        sharp corners,
        top=2pt,
        bottom=2pt,
    },
    title=\textbf{定理 #1}
}

% 蓝色提示框样式
\newtcolorbox{hintbox}{
    enhanced,
    colback=cyan!10!white,       % 背景颜色
    colframe=blue!75!black,      % 边框颜色
    fonttitle=\bfseries,
    attach boxed title to top left={yshift=-2mm, xshift=3mm},
    boxed title style={
        colback=blue!75!black,   % 标题框背景颜色
        sharp corners,
        top=2pt,
        bottom=2pt,
    },
    title=\textbf{提示}
}

% --- 设置页眉页脚 ---
\pagestyle{fancy}
\fancyhf{} % 清空所有页眉页脚
\fancyhead[R]{1 考点知识汇总} % 页眉右侧
\fancyfoot[C]{- \thepage -} % 页脚中间
\renewcommand{\headrulewidth}{0pt} % 去掉页眉线
\renewcommand{\footrulewidth}{0pt} % 去掉页脚线


\begin{document}
\thispagestyle{fancy} % 应用页眉页脚样式到第一页

\begin{center}
    \section*{1 考点知识汇总}
\end{center}

\noindent
数学部分近年考察常常与函数,导数,概率,几何,集合,数论等知识综合考察,新高考中难度有较多提升,本次讲义主要从数列与函数综合角度讨论数列应用。

\subsection*{1.1 典型数列递推求通项}

\begin{purpletheorem}{1.1 线性递推关系}
线性递推关系:$a_n = pa_{n-1} + qa_{n-2} \ (n \ge 3)$ 其特征方程为$\lambda^2 - p\lambda - q = 0$ 根据特征方程的解,有三种情况:
\begin{enumerate}
    \item[(1)] $\lambda_1 \neq \lambda_2$,其通项为:$a_n = A\lambda_1^n + B\lambda_2^n$,其中 $A, B$ 为待定系数,根据初始条件 $a_1, a_2$ 进行求解。
    \item[(2)] $\lambda_1 = \lambda_2 = \lambda$,其通项为 $a_n = (A+Bn)\lambda^n$。
    \item[(3)] 将此方程无实数解,即根为共轭复根时,$a_n = r^n(C\cos(n\theta) + D\sin(n\theta))$,其中 $C, D$ 为实数,根据 $a_1, a_2$ 求解,特别地,当 $r=1$ 时,$\{a_n\}$ 是周期数列,周期是 $\displaystyle \frac{2\pi}{\theta}$。
\end{enumerate}
\end{purpletheorem}

\subsection*{1.2 分式递推关系}

\begin{purpletheorem}{1.2 分式递推关系}
分式递推关系 $a_{n+1} = \displaystyle\frac{ba_n+c}{da_n+e}$,这里我们利用不动点方程 ($x = f(x)$,$x$ 称为 $f(x)$ 的不动点) $x = \displaystyle\frac{bx+c}{dx+e} \implies dx^2 + (e-b)x - c = 0$,同样这里有三种情况:
\begin{enumerate}
    \item[(1)] 不动点方程有两个不相等实根 $x_1 \neq x_2$,$b_n = \displaystyle\frac{a_n - x_1}{a_n - x_2}$,$\{b_n\}$ 是等比数列,公比可以根据 $b_1, b_2$ 计算。
    \item[(2)] 不动点方程有两个相等实根 $x_1 = x_2 = x_0$,$c_n = \displaystyle\frac{1}{a_n - x_0}$,$\{c_n\}$ 是等差数列,公差根据 $c_1, c_2$ 计算。
    \item[(3)] 不动点有两个共轭复根的时候,$b_n = \displaystyle\frac{a_n - x_1}{a_n - x_2}$ 是复数等比数列,可以利用 $b_1, b_2$ 求出复公比 $k = r(\cos\theta + i\sin\theta)$,可得:$b_n = b_1 k^{n-1}$,再解出 $a_n$ 即可,特别地,当 $r=1$ 时,$\{b_n\}$ 为周期数列,周期是 $\displaystyle\frac{2\pi}{\theta}$。
\end{enumerate}
\end{purpletheorem}

\subsection*{1.3 典型数列求和}

\begin{bluetheorem}{1.3 平方数数列前n项和}
平方数数列前 $n$ 项和:
$$ S_n = \sum_{k=1}^{n} k^2 = \frac{n(n+1)(2n+1)}{6} $$
\end{bluetheorem}

\begin{hintbox}
可以采用裂项法:$(k+1)^3 - k^3 = 3k^2 + 3k + 1$
$$ \sum_{k=1}^{n}((k+1)^3 - k^3) = (n+1)^3 - 1 = 3\sum_{k=1}^{n}k^2 + 3\sum_{k=1}^{n}k + n $$
\end{hintbox}

\end{document}