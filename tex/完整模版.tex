\documentclass{ctexart}

\usepackage{amsmath}
\usepackage{amssymb}
\usepackage{tcolorbox}
\usepackage{enumitem}
\usepackage{tikz}

% 引入 tcolorbox 库
\tcbuselibrary{skins, breakable}
\usetikzlibrary{shapes.misc}

% 自定义颜色
\definecolor{myred}{RGB}{237, 28, 36}
\definecolor{mypink}{RGB}{254, 242, 242}
\definecolor{mygreen}{RGB}{76, 175, 80}
\definecolor{mylightgreen}{RGB}{232, 245, 233}
\definecolor{myblue}{RGB}{33, 150, 243}
\definecolor{mylightblue}{RGB}{227, 242, 253}
\definecolor{mygray}{RGB}{158, 158, 158}
\definecolor{mylightgray}{RGB}{248, 248, 248}

% 公式盒子(红色标签)
\newtcolorbox{formulaBox}[1]{
    enhanced,
    breakable,
    colback=mypink,
    colframe=myred,
    boxrule=0.8pt,
    arc=2pt,
    fonttitle=\bfseries\color{white},
    colbacktitle=myred,
    attach boxed title to top left={
        xshift=10pt,
        yshift=-0.5\tcboxedtitleheight
    },
    boxed title style={
        arc=2pt,
        boxrule=0pt,
        overlay={
            \path[fill=myred] 
                (frame.north east) -- ++(8pt,-6pt) -- (frame.south east) -- cycle;
        }
    },
    title={#1}
}

% 例题盒子(绿色标签)
\newtcolorbox{exampleBox}[1]{
    enhanced,
    breakable,
    colback=mylightgreen,
    colframe=mygreen,
    boxrule=0.8pt,
    arc=2pt,
    fonttitle=\bfseries\color{white},
    colbacktitle=mygreen,
    attach boxed title to top left={
        xshift=10pt,
        yshift=-0.5\tcboxedtitleheight
    },
    boxed title style={
        arc=2pt,
        boxrule=0pt,
        overlay={
            \path[fill=mygreen] 
                (frame.north east) -- ++(8pt,-6pt) -- (frame.south east) -- cycle;
        }
    },
    title={#1}
}

% 解析盒子(灰色标签)
\newtcolorbox{analysisBox}[1]{
    enhanced,
    breakable,
    colback=mypink,
    colframe=mygray,
    boxrule=0.8pt,
    arc=2pt,
    fonttitle=\bfseries\color{white},
    colbacktitle=mygray,
    attach boxed title to top left={
        xshift=10pt,
        yshift=-0.5\tcboxedtitleheight
    },
    boxed title style={
        arc=2pt,
        boxrule=0pt,
        overlay={
            \path[fill=mygray] 
                (frame.north east) -- ++(8pt,-6pt) -- (frame.south east) -- cycle;
        }
    },
    title={#1}
}

% 测验盒子(蓝色标签)
\newtcolorbox{quizBox}[1]{
    enhanced,
    breakable,
    colback=mylightblue,
    colframe=myblue,
    boxrule=0.8pt,
    arc=2pt,
    fonttitle=\bfseries\color{white},
    colbacktitle=myblue,
    attach boxed title to top left={
        xshift=10pt,
        yshift=-0.5\tcboxedtitleheight
    },
    boxed title style={
        arc=2pt,
        boxrule=0pt,
        overlay={
            \path[fill=myblue] 
                (frame.north east) -- ++(8pt,-6pt) -- (frame.south east) -- cycle;
        }
    },
    title={#1}
}

\begin{document}

% 公式盒子示例
\begin{formulaBox}{公式}
    \begin{enumerate}[label=\Roman*., labelindent=0pt, leftmargin=*, itemsep=8pt]
        \item 铅直渐近线 $\lim\limits_{x \to x_0^-} f(x) = \pm\infty$ 或 $\lim\limits_{x \to x_0^+} f(x) = \pm\infty$,则称 $l: x=x_0$ 为 $f(x)$ 的铅直渐近线.
        
        \item 水平渐近线 $\lim\limits_{x \to -\infty} f(x) = y_0$ 或 $\lim\limits_{x \to +\infty} f(x) = y_0$,则称 $l: y=y_0$ 为 $f(x)$ 的水平渐近线.
        
        \item 斜渐近线 $\lim\limits_{x \to \infty} [f(x) - (kx+b)] = 0$,则称 $l: y=kx+b$ 为 $f(x)$ 的斜渐近线. 其中:
        \begin{center}
             $\lim\limits_{x \to \infty} [f(x) - kx] = b \Rightarrow \lim\limits_{x \to \infty} \left[\frac{1}{x}(f(x) - kx)\right] = 0 \Rightarrow \lim\limits_{x \to \infty} \frac{f(x)}{x} = k.$
        \end{center}
    \end{enumerate}
\end{formulaBox}

\vspace{1em}

% 例题盒子示例
\begin{exampleBox}{例题 71}
    (2025 新 TS 联考) 设曲线 $\Gamma: x^2(x-y) = 2$,则下列叙述正确的是:\underline{\hspace{3cm}}。

    \begin{enumerate}[label=\Alph*., labelindent=0pt, leftmargin=*, itemsep=4pt]
        \item 曲线 $\Gamma$ 的图象仅在第一、三象限内.
        \item 曲线 $\Gamma$ 的渐近线为 $y = x$ 和 $y$ 轴.
        \item 曲线 $\Gamma$ 与曲线 $L: y^2(y-x) = 2$ 关于直线.
        \item 曲线 $\Gamma$ 与圆 $O: x^2 + y^2 = 2$ 交于 $A, B$ 两点,则线段 $AB$ 的弦长为 $\sqrt{2} + 1$.
    \end{enumerate}
\end{exampleBox}

\vspace{1em}

% 解析盒子示例
\begin{analysisBox}{解析}
    \textbf{渐近线}
    
    易见 $\Gamma: y = x - \frac{2}{x^2}$,且 $x \neq 0, y \neq x$,所以 $\Gamma$ 无渐近线。
    
    若 $x \to \pm\infty$,则 $y - x = -\frac{2}{x^2} \to 0$,即 $y \to x$。故 $y = x$ 为 $\Gamma$ 的斜渐近线。
    
    \textbf{单调性}
    
    易有 $y' = 1 + \frac{4}{x^3}$,可见 $x > 0$ 时,$\Gamma$ 单调递增。$\left(-\infty, -\sqrt[3]{4}\right)$ 时,$\Gamma$ 单调递减;
    
    且 $x < 0$ 时,$\Gamma \leq -2\sqrt{3} - 2^{1-1} = -2 - \sqrt{2}$。
    
    \textbf{交点}
    
    由于 $x - 2$ 与 $x > 0$ 时,$\Gamma$ 的图象关于 $x = 0$ 时的无关,且 $A(x_1, y_1), B(x_2, y_2)$
    
    $k_{AB} = \frac{y_2 - y_1}{x_2 - x_1} = 1 + \frac{2(x_2 + x_1)}{(x_1 x_2)^2} > 1 + \frac{4}{\sqrt{(x_1 x_2)^3}} > 1 + \sqrt{2}$。
\end{analysisBox}

\vspace{1em}

% 测验盒子示例
\begin{quizBox}{测验 2Q}
    分别求三种曲线 $y = x + \frac{1}{x}$,$y = x - \frac{1}{x}$,$\frac{x^2}{a^2} - \frac{y^2}{b^2} = 1$ 的渐近线。
\end{quizBox}

\end{document}
